\chapter{Trabalhos Relacionados}
\label{chap.related-work}

Neste capítulo serão mostradas técnicas e pesquisas que estão sendo desenvolvidas no que diz respeito à virtualização e migração de processos. Serão apresentados trabalhos relacionados, bem como serão evidenciadas as semelhanças e diferenças com o presente trabalho.

Grande parte das pesquisas relacionadas a migração estão inseridas em ambientes \cloud. Nesses casos, os esforços estão voltados para redução do tempo total de migração, diminuição do \downtime~\cite{migration-linux-conteiners,clark2005live} e exploração das vantagens que a migração de processos oferece nesses ambientes computacionais. Entre elas podem-se citar:
\begin{enumerate}[label=(\roman*)]
    \item Balanceamento de carga~\cite{live-vm-migration-techniques,ada-things};
    \item Tolerância a falhas~\cite{fernando2019live};
    \item Gerenciamento do consumo de energia~\cite{aldossary2018performance};
    \item Compartilhamento de recursos; e
    \item Manutenção de sistemas sem interrupções~\cite{live-vm-migration-techniques,ada-things}.
\end{enumerate}

\todo[inline]{Deveria ter 2--3 frases para detalhar cada um dos trabalhos citados: qual é o problema que o trabalho ataca? qual é a solução proposta para atacar o problema? quais são os principais resultados obtidos?}

Os autores costumam aliar uma ou mais das características citadas acima para desenvolverem suas pesquisas. Por exemplo:
% \begin{enumerate}[label=(\roman*)]
    % \item 
    Aliando-se a transparência de localidade de execução de aplicações com a manutenção destas sem sua suspenção é possível explorar algorítmos ou modelos para melhor posicionar as aplicações na rede com o objetivo de atender a maior quantidade de usuários e de maneira mais eficiente e sem que o sistema precise ser desligado~\cite{live-migration-sdn}. Ademais, esses modelos podem ser especializados para tipos específicos de aplicações, como \iot~\cite{ada-things}.
% \end{enumerate}

Em contraste com os trabalhos apresentados, nossa proposta não está inserido em ambientes \cloud. Seu foco está na migração de processos entre \clusters em um mesmo \chip, em um sistema computacional restritivo que restringe as técnicas possíveis a serem utilizadas. Além disso, nosso trabalho se difere dos demais trabalhos por explorar a virtualização e migração de processos inserida no contexto dos \lws e não apenas buscando otimizar algoritmos específicos, \eg \livemigration.