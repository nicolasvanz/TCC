\imprimircapa%
%\imprimirfolhaderosto*
% Atenção! esse \protect é importante
%\protect\incluirfichacatalografica{ficha.pdf}
%\imprimirfolhadecertificacao

%m\begin{dedicatoria}
  Dedico este trabalho aos meus pais,
  
  aos demais membros da família e aos meus amigos.
\end{dedicatoria}
%\begin{agradecimentos}
  Agradeço a Deus pela minha vida.
  
  Agradeço aos meus pais e minha família, que sempre se me motivaram e confiaram na minha capacidade de superar os obstáculos da vida.
  
  Agradeço a todos que de alguma forma contribuiram no desenvolvimento deste trabalho e me auxiliaram na jornada de me tornar um profissional mais capaz. Em especial, agradeço ao meu professor orientador Márcio Bastos Castro, ao meu coorientador João Vicente Souto e aos demais colegas de projeto. 
  
  Agradeço ao Conselho Nacional de Desenvolvimento Científico e Tecnológico pelo auxílio através do Programa Institucional de Bolsas de Iniciação Cientifica (PIBIC).
  
  Agradeço aos meus amigos de curso pela convivência intensa e companheirismo durante os últimos anos.
  
\end{agradecimentos}
%\begin{epigrafe}
    So we keep asking, over and over, until a handful \\
    of earth stops our mouths — but is that an answer?

    (Heine, H., The Lazarus Poems, 1851)
\end{epigrafe}
\begin{resumo}[Resumo]
  A classe de processadores \lw surgiu para prover um alto grau de paralelismo e eficiência energética. Contudo, o desenvolvimento de aplicações para esses processadores enfrenta diversos problemas de programabilidade provenientes de suas peculiaridades arquitetônicas. Especialmente, o gerenciamento de processos precisa mitigar problemas provenientes das pequenas memórias locais e da falta de um suporte robusto para virtualização. Nesse contexto, este trabalho visa desenvolver a funcionalidade de migração de processos em um \os distribuído para \lws através de uma abordagem de virtualização leve baseada em contêineres. Particularmente, este trabalho está incluído no projeto \nanvix, um \os distribuído baseado em uma abordagem \multikernel de código aberto projetado para \lws. Os resultados experimentais mostram que a virtualização impactou positivamente o desempenho do \so. Houve aumento de desempenho no subsistema de \threads e redução de desvios, faltas na \dcache e faltas na \icache. Os processos puderam ser transferidos entre \clusters do processador em um \downtime que varia entre 19~ms e 101~ms, dependendo da quantidade de recursos utilizados.

  % Atenção! a BU exige separação através de ponto (.). Ela recomenda de 3 a 5 keywords
  \vspace{\baselineskip} 
  \textbf{Palavras-chave:} Lightweight Manycores. Sistemas Operacionais. Migração de Processos. Virtualização. Conteinerização
\end{resumo}

\begin{abstract}
  
The lightweight manycore processor class emerged to provide a high degree of parallelism and energy efficiency. However, developing applications for these processors faces various programmability issues stemming from their architectural peculiarities. Particularly, process management needs to mitigate problems arising from small local memories and the lack of robust virtualization support. In this context, this work aims to develop a process migration functionality in a distributed operating system for lightweight manycores through a lightweight container based virtualization approach. Specifically, this work is part of the Nanvix project, which is an open-source  distributed operating system based on a multikernel approach designed for lightweight manycores. Experimental results show that virtualization positively impacted the operating system's performance. There was an increase in performance in the thread subsystem and a reduction in branches, in data cache misses and instruction cache misses. The processes were able to be transferred between processor clusters with a down time ranging from 19~ms to 101~ms, depending on the amount of resources used.

  \vspace{\baselineskip} 
  \textbf{Keywords:} Lightweight Manycores. Operating Systems. Process Migration. Virtualization. Containerization
\end{abstract}

\listoffigures*
    
% Lista para ambiente algorithm
% \listofalgorithms*

% \begin{listadesimbolos}
%   $\gets$   & Atribuição \\
%   $\exists$   & Quantificação existencial \\
%   $\rightarrow$   & Implicação \\
%   $\wedge$   & E lógico \\
%   $\vee$   & Ou lógico \\
%   $\neg$   & Negação lógica \\
%   $\mapsto$   & Mapeia para \\
%   $\sqsubseteq$   & Subclasse (em ontologias) \\
%   $\subseteq$   & Subconjunto: $\forall x\;.\; x \in A \rightarrow x \in B$ \\
%   $\langle\ldots\rangle$ & Tupla \\
%   $\forall$   & Quantificação universal \\
%   mmmmm & Nenhum sentido, apenas estou aqui para demonstrar a largura máxima dessas colunas. Ao abrir o ambiente \texttt{listadesimbolos}, pode-se fornecer um argumento opcional indicando a largura da coluna da esquerda (o default é de 5em): \texttt{\textbackslash{}begin\{listadesimbolos\}[2cm] .... \textbackslash{}end\{listadesimbolos\}} \\
% \end{listadesimbolos}

\tableofcontents*%