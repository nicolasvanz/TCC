\chapter{Introdução}
    \label{ch:intro}
    
    Bem-vindo ao template no overleaf classe \texttt{ufsc-thesis-rn46-2019}. Essa
    classe é um conjunto de customizações aplicadas à classe
    \href{https://ctan.org/pkg/abntex2}{\abnTeX} e ao pacote \texttt{abntex2cite}.
    O objetivo da classe \texttt{ufsc-thesis-rn46-2019} é simplório: adequar o
    \abnTeX{} às \href{http://portal.bu.ufsc.br/normalizacao/}{normas emitidas pela
    Biblioteca Universitária da UFSC} em sequência à
    \href{https://repositorio.ufsc.br/handle/123456789/197121}{Resolução Normativa
    nº 46/2019/CPG}.
    
    
    \section{Autores, suporte e atualizações}
    
    Essa classe foi escrita inicialmente por dois alunos do
    \href{http://ppgcc.posgrad.ufsc.br/}{PPGCC da UFSC}:
    \href{mailto:alexishuf@gmail.com}{Alexis Huf} e
    \href{mailto:gustavo.zambonin@posgrad.ufsc.br}{Gustavo Zambonin}.  
    Há o risco de esse arquivo não ser atualizado a cada \textit{pull request},
    então confira a lista de mártires no GitHub. Essa classe é mantida no
    repositório
    \href{https://github.com/alexishuf/ufsc-thesis-rn46-2019/}{alexishuf/ufsc-thesis-rn46-2019}.
    Atualizações podem ser encontradas nesse repositório. \textit{Issues} e PRs são
    bem vindos.
    
    \section{Exemplos de formatação}
    \label{sec:ex}
    
    Essa frase é verdadeira pois tem um \texttt{cite} no final \cite{turing1937}. Essa
    é mais verdadeira ainda pois tem um  \texttt{cite} duplo no
    final \cite{turing1937,dijkstra1968}. Já esta frase inofensiva usa
     \texttt{citeonline} para citar \citeonline{dijkstra1968}
    nominalmente. O trabalho de \citeonline{diffie1976} foi altamente influente
    \cite{diffie1976}. Essa outra frase cita o trabalho que \citeonline{Saleem2018}
    escreveu com outros 4 autores. Para algo completamente novo, veja um footnote
    com url\footnote{\url{http://example.org/}}
    
    Mais algumas citações de tipos específicos de documentos:
    \begin{itemize}
    \item @inproceedings: \citeonline{Ullman1989magic}, jabuti
      \cite{Ullman1989magic};
    \item @article: \citeonline{Distefano2019}, framboesa \cite{Distefano2019};
    \item @book: \citeonline{Abiteboul1995}, goiaba \cite{Abiteboul1995};
    \item @incollection: \citeonline{Forgy1989}, melancia \cite{Forgy1989};
    \item @techreport: \citeonline{rdf11}, figo \cite{rdf11}.
    \end{itemize}
    
    A lista abaixo mostra o efeito de  \texttt{autoref} com capítulos e (sub)seções.
    
    \begin{itemize}
    \item Há coisas no \autoref{ch:intro};
    \item Há coisas na \autoref{sec:stuff};
    \item Há coisas na \autoref{sec:other};
    \item Há coisas na \autoref{sec:yet-another} (\abnTeX{} come um ``sub'' intencionalmente).
    \end{itemize}
    
    Citações são feitas com o ambiente \texttt{citacao}. A BU faz
    as mesmas exigências que já são o \textit{default} na classe
    \abnTeX\footnote{O alinhamento e o filete de notas de rodapé também
    não necessitou de modificações, além do tamaho da fonte. Essa frase
    não serve a nenhum propósito além de causar uma quebra de linha para
    que o alinhamento seja avaliado.}. 
    
    \begin{citacao}
      A elaboração do trabalho de conclusão de curso em nível de mestrado
      e de doutorado na UFSC deverá atender aos critérios e procedimentos
      estabelecidos nesta resolução normativa e em diretrizes
      estabelecidas pela Pró-Reitoria de Pós-Graduação e pelos Programas
      de Pós-Graduação.
    \end{citacao}
    
    Atenção! O template da BU deixa figuras e tabelas alinhadas à esquerda. No
    entanto, o tutorial de Word disponibilizado pela BU diz que legendas e
    \emph{captions} devem respeitar o ``alinhamento da ilustração'' (e apresenta
    uma ilustração alinhada à esquerda). O tutorial explicando a ABNT mostra uma
    figura centralizada com legendas alinhadas a esquerda e com recuo até o começo
    da figura. O autor do \texttt{.cls} se exime de qualquer culpa. 
    Veja na \autoref{fig:logo} o efeito de se usar \texttt{centering}.
    
    \begin{figure}[t]
      \centering
      \caption{Logotipo da Universidade Federal de Santa Catarina.}
      \label{fig:logo}
    
      \includegraphics[width=.2\linewidth]{\jobname-logo.pdf}
      \fonte{O autor.}
    \end{figure}
    
    Algoritmos podem ser incluidos no ambiente \texttt{algorithm}. Atenção! Esse
    ambiente é apenas uma classe de \emph{float}. Logo esse ambiente apenas oferece
    a funcionalidade de colocar legendas e numerar algoritmos que serão exibidos em
    uma lista dedicada com \texttt{\textbackslash{}listofalgorithms*}. Para
    efetivamente escrever os lagoritmos considere pacotes como
    \texttt{algorithmicx}, \texttt{algorithm2e} ou \texttt{minted}.
    
    \begin{algorithm}
      \centering
      \caption{Exemplo sem sentido algum.}
    
      \fbox{
        \parbox{.6\textwidth}{
          \begin{itemize}
          \item Não use itemize como algoritmo! 
          \item Escolha um pacote como \texttt{algorithmicx}, \texttt{algorithm2e} ou \texttt{minted}.
          \end{itemize}
        }
      }
    
      \fonte{O autor.}
    \end{algorithm}
    
    \subsection{Coisas}
    \label{sec:stuff}
    Imagine alguma afirmação de alto valor científico aqui.
    
    \subsubsection{Outras coisas mais}
    \label{sec:other}
    Estudos demonstram que essa afirmação é falsa.
    
    \subsubsubsection{Ainda outras coisas mais}
    \label{sec:yet-another}
    Fazer a grama verde, como? Novamente o jogo foi perdido. Opcionalmente, tudo
    pode ser opcional. Recursos foram gastos com isso. Descubra a verdade nas
    capitalizadas.
    % Fiquei 15 minutos mais próximo da morte ao escrever isso. Você pode chegar
    % ainda mais perto se tentar entender.