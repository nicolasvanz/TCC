\begin{resumo}[Resumo]
  A classe de processadores \lw surgiu para prover um alto grau de paralelismo e eficiência energética. Contudo, o desenvolvimento de aplicações para esses processadores enfrenta diversos problemas de programabilidade provenientes de suas peculiaridades arquitetônicas. Especialmente, o gerenciamento de processos precisa mitigar problemas provenientes das pequenas memórias locais e da falta de um suporte robusto para virtualização. Nesse contexto, este trabalho visa desenvolver o suporte da migração de processos em um \os distribuído para \lws através de uma abordagem de virtualização leve baseada em contêineres. Particularmente, este trabalho está incluído no projeto \nanvix, um \os distribuído de código aberto projetado para \lws. Ao final deste trabalho espera-se melhorar o gerenciamento de processos no \nanvix, bem como abstrair e auxiliar o gerenciamento dos recursos do processador.

  % Atenção! a BU exige separação através de ponto (.). Ela recomanda de 3 a 5 keywords
  \vspace{\baselineskip} 
  \textbf{Palavras-chave:} lightweight manycores. sistemas operacionais. migração de processos. virtualização. conteinerização
\end{resumo}
