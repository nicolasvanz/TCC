\glsresetall
\chapter{Referencial Teórico}
\label{chap.background}

\mytodo{incluir base multiprocessadores/multicomputadores nesta seção}
\section{\Lws}
\label{sec.lw}



\section{\nanvix}
\mytodo{overview/threads/comunicação}
O \nanvix\footnote{Disponível em https://github.com/nanvix} é um \os distribuído e de propósito geral que busca equilibrar desempenho, portabilidade e programabilidade para \lws \cite{penna:sbesc19}. \nanvix é estruturado em 3 camadas de \kernel. São elas:
\mytodo{adicionar figura da hal, microkernel e multikernel do nanvix}
\begin{description}
    \item [\nanvix \hal]
         é a camada mais baixa que abstrai e provê o gerenciamento dos recursos de \hardware sobre uma visão comum \cite{penna:hal}. Entre eles estão: \cores, \tlbs, \cache, \mmu, \noc, interrupções, memória virtual, recursos de \io. De maneira geral, esta camada provê visões a nivel de \core, \cluster e comunicação/sincronização entre \clusters \cite{penna:thesis}
    
    \item [\nanvix \Assymetric \Microkernel]
        é a camada intermediária que provê gerenciamento de recursos e os serviços mínimos de um \os em um \cluster. Entre esses serviços se encontram a comunição intercluster, gerenciamento de \threads e memória, controle de acesso à memória e interface para chamadas de sistema. As chamadas de sistema podem ser executadas localmente, caso acessem dados \rdo ou alterem estruturas internas do \core, ou remotamente pelo \mcore que atende à requisição e libera o \score requisitante ao seu término \cite{penna:thesis}. Essa característica adjetiva o \microkernel como assimétrico.
    
    \item [\nanvix \Multikernel]
        é a camada superior que provê os serviços de um \os e dispõe uma visão a nível do processador em si. Os serviços são hospedados em \clusters \ie isolados das aplicações de usuário e atendem as requisições vindas dos processos de usuário através de um modelo cliente-servidor. As requisições e respostas são enviadas/recebidas através de passagem de mensagem via \noc. Os serviços dessa camada podem ser entendidos como fontes de informação que mantém a execução dos processos consistentes no processador. Neles estão presentes mecanismos de \spawn de processos e obtenção de nomes lógicos dos processos (a fim de localizá-los para comunicação), por exemplo.
\end{description}

\mytodo{continuar daqui}

Em sua abordagem original, os processos no \nanvix são estáticos, \ie cada \cluster possui apenas um processo. Desse modo, uma vez que o processo inicia sua execução em um \cluster, este finalizará a execução no mesmo \cluster. 
Isso torna o processo dependente do \cluster que o executa \eg a comunicação entre processos está atrelada aos \clusters nos quais os processos são executados e não aos processos em si. A falta de mobilidade dos processos nesse modelo pode trazer sobrecargas ao processador e afeta o suporte a multi-aplicação. Por exemplo, a comunicação entre \clusters próximos é mais rápida e resulta em menor consumo energético do processador. Sendo assim, melhorar a mobilidade e a disposição dos processos no processador \ie viabilizar a migração de processos entre \clusters, possibilitaria melhorar o gerenciamento dos recursos do mesmo.

\section{Virtualização e Conteinerização}
\label{sec.virtualização}
