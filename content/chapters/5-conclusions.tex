\glsresetall
\chapter{Conclusões}
\label{chap.conclusions}

Neste trabalho foi explorado um modelo de virtualização leve baseada em contêineres que considera as restrições arquiteturais dos \lws, se adaptando as suas restrições, principalmente relacionadas à memória, visando de melhorar a mobilidade de processos em um \os distribuído. Isso com o objetivo de aumentar a eficiência dos processadores \lws e acentuar ainda mais suas principais características: alto poder de processamento e economia de energia.

Os resultados mostraram que o isolamento das dependências de um processo aumentaram o desempenho de operações do \kernel e suportaram de fato a migração de processos de forma eficiente. Como trabalhos futuros, pretende-se 
\begin{enumerate}[label=(\roman*)]
    \item Ampliar a virtualização, englobando outros subsistemas do \nanvix;
    \item Habilitar a execução simultânea de múltiplas aplicações no processador e sua proteção.
\end{enumerate}