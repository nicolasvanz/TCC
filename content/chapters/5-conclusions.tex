\glsresetall
\chapter{Conclusões}
\label{chap.conclusions}

Neste trabalho, foi explorado um modelo de virtualização leve baseado em contêineres que considera as restrições arquiteturais dos \lws, adaptando-se as suas restrições, principalmente relacionadas à memória. A virtualização proposta visa melhorar a mobilidade e gerenciamento de processos para \lws no contexto de em um \os distribuído, o \nanvix.
%

Os resultados mostraram que a virtualização nesses ambientes é possível, bem como a migração de processos entre os \clusters do processador. A migração não afeta significativamente o sistema de comunicação, sendo possível realizar múltiplas migrações simultaneamente. Neste contexto, a conteinerização exerceu o papel principal ao evitar o envio de dados redundantes relativos ao \kernel e a melhor organizar os dados internos do \kernel e do usuário. 

A migração provocou um \downtime que varia entre 19~ms e 101~ms, dependendo da quantidade de recursos utilizados. O isolamento das dependências de um processo aumentaram o desempenho de operações do \kernel na execução normal do \so, diminuindo a quantidade de desvios, faltas na \dcache e faltas na \icache em 7,8~\%, 5,8~\% e 6,45~\%, respectivamente.

\section{Trabalhos Futuros}
Como trabalhos futuros, pretende-se:

\begin{enumerate}[label=(\roman*)]
    \item Ampliar a virtualização, englobando o subsistema de comunicação;
    \item Habilitar a execução simultânea de múltiplas aplicações no processador e sua proteção;
    \item Implementar um escalonador no \nanvix que considere a melhor distribuição de processos entre os \clusters;
    \item Implementar um sistema de \checkpointing que torne possível guardar estados de processos em disco;
\end{enumerate}

