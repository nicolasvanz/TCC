\chapter{Referencial Teórico}
\label{chap.background}

\section{Multiprocessadores}
\label{sec.multiprocessadores}

\section{Multicomputadores}
\label{sec.multicomputadores}

\todo{incluir base multiprocessadores/multicomputadores nesta seção}
\section{\Lws}
\label{sec.lw}

\section{\nanvix}
\todo{overview/threads/comunicação}
O \nanvix\footnote{Disponível em https://github.com/nanvix} é um \os distribuído e de propósito geral que busca equilibrar desempenho, portabilidade e programabilidade para \lws  \cite{penna:sbesc19}. \nanvix é estruturado em 3 camadas de \kernel. São elas:
\begin{enumerate}[label=(\roman*)]
    \item \nanvix \hal: é a camada mais baixa que abstrai os recursos de \hardware sobre uma visão comum;
    
    \item \nanvix \microkernel: é a camada intermediária que provê gerenciamento de recursos e os serviços mínimos de um \os em um \cluster;
    
    \item \nanvix \multikernel: é a camada superior que provê os serviços de um \os. Os serviços atendem as requisições vindas dos processos de usuário através de um modelo cliente-servidor;
\end{enumerate}
    
Em sua abordagem original, os processos no \nanvix são estáticos, \ie cada \cluster possui apenas um processo. Desse modo, uma vez que o processo inicia sua execução em um \cluster, este finalizará a execução no mesmo \cluster. 
Isso torna o processo dependente do \cluster que o executa \eg a comunicação entre processos está atrelada aos \clusters nos quais os processos são executados e não aos processos em si. A falta de mobilidade dos processos nesse modelo pode trazer sobrecargas ao processador e afeta o suporte a multi-aplicação. Por exemplo, a comunicação entre \clusters próximos é mais rápida e resulta em menor consumo energético do processador. Sendo assim, melhorar a mobilidade e a disposição dos processos no processador \ie viabilizar a migração de processos entre \clusters, possibilitaria melhorar o gerenciamento dos recursos do mesmo.

\todo{trocar por entrada no glossário}
\section{\nanvix \textit{Hardware Abstraction Layer} (HAL)}
\label{sec.nanvixhal}

\subsection{\nanvix \microkernel}
\label{sec.nanvixmicrokernel}

\subsection{\nanvix \multikernel}
\label{sec.nanvixmultikernel}

\section{Virtualização e Conteinerização}
\label{sec.virtualização}
