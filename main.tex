\documentclass[]{lapesd-thesis}

% Outros \usepackage{}
\usepackage{todonotes}
\usepackage{svg}

%%%%%%%%%%%%%%%%%%%%%%%%%%%%%%%%%%%%%%%%%%%%%%%%%%%%%%%%%%%%%%%%%%%%
%%% Configurações da classe (dados do trabalho)                  %%%
%%%%%%%%%%%%%%%%%%%%%%%%%%%%%%%%%%%%%%%%%%%%%%%%%%%%%%%%%%%%%%%%%%%%

% Informações para capa e folha de rosto/certificacao
\titulo{Virtualização e Migração de Processos em um Sistema Operacional Distribuído para Lightweight Manycores}
\autor{Nicolas Vanz}
\data{\today} % ou \today
%\tese % ou \dissertacao
\titulode{Bacharel em Ciência da Computação}
\orientador{Prof. Márcio Bastos Castro, Dr.}
\coorientador{João Vicente Souto, Me.}

%%% Atenção! No caso de TCC, além de usar \tcc, outros comandos devem ser 
%%% fornecidos:
\tcc
\departamento{Departamento de Informática e Estatística}
\curso{Ciência da Computação}
\titulode{Bacharel em Ciência da Computação}
% %% Para TCCs, orientadores e coorientadores podem ser externos, logo a
% %% BU exige que sua afiliação seja explicitada. Por padrão, assume-se
% %% UFSC. Você pode alterar a afiliação com os comandos abaixo:
% \afiliacaoorientador{Universidade Federal de Santa Catarina}
% \afiliacaocoorientador{Universidade Federal da Terra de Ninguém}

% Membros da banca e coordenador
% As regras da BU agora exigem que Dr. apareça depois do nome
% Dica: para gerar Profᵃ. use Prof\textsuperscript{a}.
% Dica 2: para feminino use \orientadora e \coorientadora
%\membrobanca{Prof. Valerie Béranger, Dr.}{Universidade Federal de Santa Catarina}
%\membrobanca{Prof. Mordecai Malignatus, Dr.}{Universidade Federal de Santa Catarina}
%\membrobanca{Prof. Huifen Chan, Dr.}{Universidade Federal de Santa Catarina}
\coordenador{Prof. Renato Cislaghi, Dr.}


% Ativa indíce remissivo. Precisa estar aqui, não funciona no .cls nem no
% \BeforeBeginDocument{}
\makeindex
% Carrega definições dos acrônimos e do glossário. Isso precisa ser feito antes
% do \begin{document}
%==============================================================================
% Computer Architecture
%=============================================================================
\newcommand{\chips}{\textit{chips}\xspace}
\newcommand{\core}{\textit{core}\xspace}
\newcommand{\cores}{\textit{cores}\xspace}
\newcommand{\exascale}{\textit{exascale}\xspace}
\newcommand{\manycore}{\textit{manycore}\xspace}
\newcommand{\manycores}{\textit{manycores}\xspace}
\newcommand{\lw}{\textit{lightweight manycore}\xspace}
\newcommand{\lws}{\textit{lightweight manycores}\xspace}
\newcommand{\Lw}{\textit{Lightweight manycore}\xspace}
\newcommand{\Lws}{\textit{Lightweight manycores}\xspace}
\newcommand{\multicore}{\textit{multicore}\xspace}
\newcommand{\multicores}{\textit{multicores}\xspace}

% Manycores
\newcommand{\scc}{Intel Single-Cloud Computer\xspace}
\newcommand{\xeonphi}{Intel Xeon Phi\xspace}
\newcommand{\tilegx}{Tilera TILE-Gx100\xspace}
\newcommand{\tilepro}{Tilera TILE64\xspace}
\newcommand{\mppa}{Kalray MPPA-256\xspace}
\newcommand{\taihulight}{Sunway SW26010\xspace}
\newcommand{\epiphany}{Adapteva Epiphany\xspace}
\newcommand{\optimsoc}{OpTiMSoC\xspace}
\newcommand{\hero}{HERO\xspace}
\newcommand{\arm}{ARM Cortex-A\xspace}
\newcommand{\riscv}{RISC-V\xspace}

% Architectures
\newcommand{\intel}{x86\xspace}
\newcommand{\openrisc}{OpenRISC\xspace}
\newcommand{\bostan}{Bostan\xspace}

% Instruction Set Architectures
\newacronym{isa}{ISA}{Instruction Set Architecture}
	\newcommand{\isa}{\gls{isa}\xspace}
\newacronym{risc}{RISC}{Reduced Instruction Set Computer}
	\newcommand{\risc}{\gls{risc}\xspace}
\newacronym{vliw}{VLIW}{Very Long Instruction Word}
	\newcommand{\vliw}{\gls{vliw}\xspace}

% Taxonomy
\newacronym{sisd}{SISD}{\textit{Single Instruction Single Data}}
	\newcommand{\sisd}{\gls{sisd}\xspace}
\newacronym{simd}{SIMD}{\textit{Single Instruction Multiple Data}}
	\newcommand{\simd}{\gls{simd}\xspace}
\newacronym{misd}{MISD}{\textit{Multiple Instruction Single Data}}
	\newcommand{\misd}{\gls{misd}\xspace}
\newacronym{mimd}{MIMD}{\textit{Multiple Instruction Multiple Data}}
	\newcommand{\mimd}{\gls{mimd}\xspace}
\newacronym{uma}{UMA}{Uniform Memory Access}
	\newcommand{\uma}{\gls{uma}\xspace}
\newacronym{numa}{NUMA}{Non-Uniform Memory Access}
	\newcommand{\numa}{\gls{numa}\xspace}
\newacronym{norma}{NoRMA}{No Remote Memory Access}
	\newcommand{\norma}{\gls{norma}\xspace}
\newacronym{amp}{AMP}{Asymmetric Multi-Processing}
	\newcommand{\amp}{\gls{amp}\xspace}
\newacronym{cpu}{CPU}{\textit{Central Processing Unit}}
	\newcommand{\cpu}{\gls{cpu}\xspace}
	\newcommand{\cpus}{\glspl{cpu}\xspace}
\newacronym{gpu}{GPU}{Graphics Processing Unit}
	\newcommand{\gpu}{\gls{gpu}\xspace}
	\newcommand{\gpus}{\glspl{gpu}\xspace}
\newacronym{fpga}{FPGA}{Field Programmable Gate Array}
	\newcommand{\fpga}{\gls{fpga}\xspace}
	\newcommand{\fpgas}{\glspl{fpga}\xspace}
\newacronym{soc}{SoC}{System-on-a-Chip}
	\newcommand{\soc}{\gls{soc}\xspace}
	\newcommand{\socs}{\glspl{soc}\xspace}
\newacronym{mpsoc}{MPSoC}{\textit{Multiprocessor System-on-Chip}}
	\newcommand{\mpsoc}{\gls{mpsoc}\xspace}
	\newcommand{\mpsocs}{\glspl{mpsoc}\xspace}
\newacronym{cmp}{CMP}{Chip Multiprocessor}
	\newcommand{\cmp}{\gls{cmp}\xspace}
	\newcommand{\cmps}{\glspl{cmp}\xspace}
\newacronym{spm}{SPM}{Software-managed Scratchpad Memories}
	\newcommand{\spm}{\gls{spm}\xspace}
	\newcommand{\spms}{\glspl{spm}\xspace}
\newacronym{pmca}{PMCA}{Programmable Manycore Accelerator}
	\newcommand{\pmca}{\gls{pmca}\xspace}

% Core
\newacronym{pe}{PE}{\textit{Processing Element}}
	\newcommand{\pe}{\gls{pe}\xspace}
	\newcommand{\pes}{\glspl{pe}\xspace}
	\newacronym{rm}{RM}{\textit{Resource Manager}}
	\newcommand{\rman}{\gls{rm}\xspace}
	\newcommand{\rmans}{\glspl{rm}\xspace}

\newcommand{\cluster}{\textit{cluster}\xspace}
\newcommand{\clusters}{\textit{clusters}\xspace}
\newcommand{\Cluster}{\textit{Cluster}\xspace}
\newcommand{\Clusters}{\textit{Clusters}\xspace}
\newcommand{\iocluster}{\textit{I/O Cluster}\xspace}
\newcommand{\ioclusters}{\textit{I/O Clusters}\xspace}
\newcommand{\ccluster}{\textit{Compute Cluster}\xspace}
\newcommand{\cclusters}{\textit{Compute Clusters}\xspace}

% Memory
\newacronym{mmu}{MMU}{\textit{Memory Management Unit}}
	\newcommand{\mmu}{\gls{mmu}\xspace}
	\newcommand{\mmus}{\glspl{mmu}\xspace}

\newacronym{mpu}{MPU}{Memory Protection Unit}
	\newcommand{\mpu}{\gls{mpu}\xspace}

\newacronym{icache}{I-cache}{\textit{Instruction Cache}}
	% \newcommand{\icache}{\gls{icache}\xspace}
	\newcommand{\icache}{\cache de instruções\xspace}

\newacronym{dcache}{D-cache}{\textit{Data Cache}}
	% \newcommand{\dcache}{\gls{dcache}\xspace}
	\newcommand{\dcache}{\cache de dados\xspace}

\newacronym{rab}{RAB}{Remapping Address Block}
	\newcommand{\rab}{\gls{rab}\xspace}

\newacronym{tlb}{TLB}{\textit{Translation Lookaside Buffer}}
	\newcommand{\tlb}{\gls{tlb}\xspace}
	\newcommand{\tlbs}{\glspl{tlb}\xspace}

\newacronym{jtlb}{JTLB}{Join TLB}
	\newcommand{\jtlb}{\gls{jtlb}\xspace}
	\newcommand{\jtlbs}{\gls{jtlb}\xspace}

\newacronym{ltlb}{LTLB}{Locked TLB}
	\newcommand{\ltlb}{\gls{ltlb}\xspace}
	\newcommand{\ltlbs}{\glspl{ltlb}\xspace}

\newacronym{itlb}{ITLB}{Instruction TLB}
	\newcommand{\itlb}{\gls{itlb}\xspace}

\newacronym{dtlb}{DTLB}{Data TLB}
	\newcommand{\dtlb}{\gls{dtlb}\xspace}

\newacronym{ram}{RAM}{Random Access Memory}
	\newcommand{\ram}{\gls{ram}\xspace}

\newacronym{sram}{SRAM}{\textit{Static Random Access Memory}}
	\newcommand{\sram}{\gls{sram}\xspace}

\newacronym{dram}{DRAM}{\textit{Dynamic Random Access Memory}}
	\newcommand{\dram}{\gls{dram}\xspace}

\newacronym{spd}{SPM}{Scratchpad Memory}
	\newcommand{\spd}{\gls{spd}\xspace}
\newacronym{dma}{DMA}{Acesso Direto à Memória}
	\newcommand{\dma}{\gls{dma}\xspace}
	\newcommand{\dmas}{\glspl{dma}\xspace}

\newacronym{rma}{RMA}{Remote Memory Access}
	\newcommand{\rma}{\gls{rma}\xspace}

% Interconnects
\newacronym[longplural={\textit{Networks-on-Chip}}]{noc}{NoC}{\textit{Network-on-Chip}}
	\newcommand{\noc}{\gls{noc}\xspace}
	\newcommand{\nocs}{\glspl{noc}\xspace}
\newacronym{cnoc}{C-NoC}{\textit{Control NoC}}
    \newcommand{\cnoc}{\gls{cnoc}\xspace}
\newacronym{dnoc}{D-NoC}{\textit{Data NoC}}
	\newcommand{\dnoc}{\gls{dnoc}\xspace}

\newacronym{wan}{WAN}{\textit{Wide-Area Network}}
	\newcommand{\wan}{\gls{wan}\xspace}

%==============================================================================
% Operating Systems
%=============================================================================
\newcommand{\Multikernel}{\textit{Multikernel}\xspace}
\newcommand{\Microkernel}{\textit{Microkernel}\xspace}
\newcommand{\multikernel}{\textit{multikernel}\xspace}
\newcommand{\microkernel}{\textit{microkernel}\xspace}

\newacronym[longplural={Sistemas Operacionais}]{os}{SO}{Sistema Operacional}
	\newcommand{\os}{\gls{os}\xspace}
	\newcommand{\oss}{\glspl{os}\xspace}
\newacronym{smp}{SMP}{Symmetric Multi-Processing}
	\newcommand{\smp}{\gls{smp}\xspace}

% HAL
\newacronym{hal}{HAL}{\textit{Hardware Abstraction Layer}}
	\newcommand{\hal}{\gls{hal}\xspace}
	\newcommand{\hals}{\glspl{hal}\xspace}

\newacronym{posix}{POSIX}{\textit{Portable Operating System Interface}}
	\newcommand{\posix}{\gls{posix}\xspace}

\newacronym{moosca}{MOOSCA}{\textit{Manycore Operating System for Safety-Critical Application}}
	\newcommand{\moosca}{\gls{moosca}\xspace}

\newcommand{\fleets}{\textit{Fleets}\xspace}

\newacronym{api}{API}{Application Programming Interface}
	\newcommand{\api}{\gls{api}\xspace}
	\newcommand{\apis}{\glspl{api}\xspace}

% Kernels
\newcommand{\barrelfish}{Barrelfish\xspace}
\newcommand{\linux}{Linux\xspace}
\newcommand{\unix}{Unix\xspace}
\newcommand{\rtems}{RTEMS\xspace}
\newcommand{\bsd}{BSD\xspace}
\newcommand{\nodeos}{NodeOS\xspace}
\newcommand{\nanvix}{Nanvix\xspace}
\newcommand{\nanvixos}{Nanvix OS\xspace}
\newcommand{\nanvixhal}{\textit{Nanvix \hal}\xspace}
\newcommand{\nanvixmicrokernel}{\textit{Nanvix Microkernel\xspace}}
\newcommand{\nanvixmultikernel}{\textit{Nanvix Multikernel\xspace}}
\newacronym{fos}{FOS}{Factored Operating System}
	\newcommand{\fos}{\gls{fos}\xspace}
\newacronym{lfour}{L4}{L4 Microkernel}
	\newcommand{\lfour}{\gls{lfour}\xspace}
\newacronym{nos}{nOS}{Nano-Sized Operating System}
	\newcommand{\nos}{\gls{nos}\xspace}
\newacronym{mos}{mOS}{multi Operating System}
	\newcommand{\mos}{\gls{mos}\xspace}

\newacronym{ipc}{IPC}{\textit{Inter-Process Communication}}
	\newcommand{\ipc}{\gls{ipc}\xspace}
\newacronym{pid}{PID}{\textit{Process ID}}
	\newcommand{\pid}{\gls{pid}\xspace}
\newacronym{uts}{UTS}{\textit{Unix Timesharing System}}
	\newcommand{\uts}{\gls{uts}\xspace}
\newcommand{\user}{\textit{user\xspace}}


\newacronym{qos}{QoS}{Qualidade de Serviço}
	\newcommand{\qos}{\gls{qos}\xspace}

%==============================================================================
% High Performance Computing
%=============================================================================

\newacronym{hpc}{HPC}{High-Performance Computing}
	\newcommand{\hpc}{\gls{hpc}\xspace}

% Runtimes
\newcommand{\openmp}{OpenMP}
\newacronym{mpi}{MPI}{Message Passing Interface}
	\newcommand{\mpi}{\gls{mpi}\xspace}
\newacronym{pgas}{PGAS}{Partitioned Global Address Space}
	\newcommand{\pgas}{\gls{pgas}\xspace}

%==============================================================================
% Other
%=============================================================================

\newacronym{rmem}{RMem}{Remote Memory}
	\newcommand{\rmem}{\gls{rmem}\xspace}

\newcommand{\ie}{i.e.,\xspace}
\newcommand{\eg}{e.g.,\xspace}
\newcommand{\aka}{\textit{aka}\xspace}
\newcommand{\etal}{\textit{et al.}\xspace}

\newcommand{\sync}{\textit{Sync}\xspace}
\newcommand{\mailbox}{\textit{Mailbox}\xspace}
\newcommand{\mailboxes}{\textit{Mailboxes}\xspace}
\newcommand{\portal}{\textit{Portal}\xspace}

\newcommand{\libipc}{\textit{MPPA IPC}\xspace}
\newcommand{\libnoc}{\textit{MPPA NoC}\xspace}

\newcommand{\kernel}{\textit{kernel}\xspace}
\newcommand{\kernels}{\textit{kernels}\xspace}
\newcommand{\benchmark}{\textit{benchmark}\xspace}
\newcommand{\benchmarks}{\textit{benchmarks}\xspace}

\newcommand{\software}{\textit{software}\xspace}
\newcommand{\softwares}{\textit{softwares}\xspace}
\newcommand{\hardware}{\textit{hardware}\xspace}
\newcommand{\hardwares}{\textit{hardwares}\xspace}

\newcommand{\cache}{\textit{cache}\xspace}
\newcommand{\caches}{\textit{caches}\xspace}

\newcommand{\thread}{\textit{thread}\xspace}
\newcommand{\threads}{\textit{threads}\xspace}
\newcommand{\Threads}{\textit{Threads}\xspace}

\newacronym{iid}{i.i.d}{Independent and Identically Distributed}
	\newcommand{\iid}{\gls{iid}\xspace}

\newacronym{anova}{ANOVA}{Analysis of Variance}
	\newcommand{\anova}{\gls{anova}\xspace}

\newacronym{cow}{COW}{Copy-On-Write}
	\newcommand{\cow}{\gls{cow}\xspace}

\newacronym{flops}{FLOPS}{\textit{Floating-point Operations per Second}}
	\newcommand{\flops}{\gls{flops}\xspace}
\newacronym{watts}{W}{\textit{Watts}}
	\newcommand{\watts}{\gls{watts}\xspace}
\newacronym{darpa}{DARPA/IPTO}{Departamento de Defesa do Governo dos Estados Unidos}
	\newcommand{\darpa}{\gls{darpa}\xspace}

\newacronym{uarea}{UArea}{\textit{User Area}}
	\newcommand{\uarea}{\gls{uarea}\xspace}


\newacronym{ufsc}{UFSC}{Universidade Federal de Santa Catarina}
	\newcommand{\ufsc}{\gls{ufsc}\xspace}
\newacronym{pucminas}{PUC Minas}{Pontifícia Universidade Católica de Minas Gerais}
	\newcommand{\pucminas}{\gls{pucminas}\xspace}
\newacronym{uga}{UGA}{Universidade de Grenoble Alpas}
	\newcommand{\uga}{\gls{uga}\xspace}

\newacronym{criu}{CRIU}{\textit{Checkpoint/Restore In Userspace}}
	\newcommand{\criu}{\gls{criu}\xspace}
	
\newacronym[longplural={Máquinas Virtuais}]{vm}{VM}{Máquina Virtual}
	\newcommand{\vm}{\gls{vm}\xspace}
	\newcommand{\vms}{\glspl{vm}\xspace}

\newacronym[longplural={\textit{Virtual Machine Monitors}}]{vmm}{VMM}{\textit{Virtual Machine Monitor}}
	\newcommand{\vmm}{\gls{vmm}\xspace}
	\newcommand{\vmms}{\glspl{vmm}\xspace}

\newacronym[longplural={\textit{Microcontroller Units}}]{mcu}{MCU}{\textit{Microcontroller Unit}}
	\newcommand{\mcu}{\gls{mcu}\xspace}
	\newcommand{\mcus}{\glspl{mcu}\xspace}

\newacronym{faas}{FaaS}{\textit{Functions as a Service}}
	\newcommand{\faas}{\gls{faas}\xspace}

\newacronym{vdi}{VDI}{\textit{Virtual Desktop Infrastructure}}
	\newcommand{\vdi}{\gls{vdi}\xspace}
	
\newacronym[longplural={\textit{control groups}}]{cgroup}{cgroup}{\textit{control group}}
	\newcommand{\cgroup}{\gls{cgroup}\xspace}
	\newcommand{\cgroups}{\glspl{cgroup}\xspace}

\newacronym{lxc}{LXC}{\textit{Linux Containers}}
	\newcommand{\lxc}{\gls{lxc}\xspace}

\newacronym{lxd}{LXD}{\textit{Linux Containers Daemon}}
	\newcommand{\lxd}{\gls{lxd}\xspace}

\newacronym{iot}{IoT}{\textit{Internet of Things}}
	\newcommand{\iot}{\gls{iot}\xspace}

\newacronym{erad}{ERAD/RS}{Escola Regional de Alto Desempenho da Região Sul}
	\newcommand{\erad}{\gls{erad}\xspace}

\newacronym{ibm}{IBM}{\textit{International Business Machines Corporation}}
	\newcommand{\ibm}{\gls{ibm}\xspace}

\newcommand{\syscall}{\textit{syscall}\xspace}
\newcommand{\desktop}{\textit{desktop}\xspace}
\newcommand{\syscalls}{\textit{syscalls}\xspace}
\newcommand{\cloud}{\textit{cloud}\xspace}
\newcommand{\io}{\textit{I/O}\xspace}
\newcommand{\assymetric}{\textit{assymetric}\xspace}
\newcommand{\Assymetric}{\textit{Assymetric}\xspace}
\newcommand{\mcore}{\textit{master core}\xspace}
\newcommand{\score}{\textit{slave core}\xspace}
\newcommand{\scores}{\textit{slave cores}\xspace}
\newcommand{\so}{\os}
\newcommand{\sos}{\oss}
\newcommand{\rdo}{\textit{read-only}\xspace}
\newcommand{\spawn}{\textit{spawn}\xspace}
\newcommand{\singlecores}{\textit{single-cores}\xspace}
\newcommand{\singlecore}{\textit{single-core}\xspace}
\newcommand{\chip}{\textit{chip}\xspace}
\newcommand{\script}{\textit{script}\xspace}
\newcommand{\freeze}{\textit{freeze}\xspace}
\newcommand{\unfreeze}{\textit{unfreeze}\xspace}
\newcommand{\buffer}{\textit{buffer}\xspace}
\newcommand{\buffers}{\textit{buffers}\xspace}
\newcommand{\task}{\textit{task}\xspace}
\newcommand{\tasks}{\textit{tasks}\xspace}
\newcommand{\Tasks}{\textit{Tasks}\xspace}
\newcommand{\dispatcher}{\textit{Dispatcher}\xspace}
\newcommand{\hotmigration}{\textit{hot migration}\xspace}
\newcommand{\Hotmigration}{\textit{Hot Migration}\xspace}
\newcommand{\Daemon}{\textit{Daemon}\xspace}
\newcommand{\daemon}{\textit{daemon}\xspace}
\newcommand{\daemons}{\textit{daemons}\xspace}
\newcommand{\libnanvix}{\textit{libnanvix}\xspace}
\newcommand{\ulibc}{\textit{ulibc}\xspace}
\newcommand{\downtime}{\textit{downtime}\xspace}
\newcommand{\Downtime}{\textit{Downtime}\xspace}
\newcommand{\livemigration}{\textit{live migration}\xspace}
\newcommand{\checkpoints}{\textit{checkpoints}\xspace}
\newcommand{\checkpoint}{\textit{checkpoint}\xspace}
\newcommand{\checkpointing}{\textit{checkpointing}\xspace}
\newcommand{\switches}{\textit{switches}\xspace}
\newcommand{\firewalls}{\textit{firewalls}\xspace}
\newcommand{\mainframe}{\textit{mainframe}\xspace}
\newcommand{\mainframes}{\textit{mainframes}\xspace}
\newcommand{\hypervisor}{\textit{hypervisor}\xspace}
\newcommand{\hypervisors}{\textit{hypervisors}\xspace}
\newcommand{\design}{\textit{design}\xspace}
\newcommand{\coldmigration}{\textit{cold migration}\xspace}
\newcommand{\Coldmigration}{\textit{Cold migration}\xspace}
\newcommand{\precopy}{\textit{pre-copy}\xspace}
\newcommand{\precopymigration}{\textit{pre-copy migration}\xspace}
\newcommand{\Precopymigration}{\textit{Pre-Copy migration}\xspace}
\newcommand{\postcopy}{\textit{post-copy}\xspace}
\newcommand{\postcopymigration}{\textit{post-copy migration}\xspace}
\newcommand{\Postcopymigration}{\textit{Post-Copy migration}\xspace}
\newcommand{\nonlivemigration}{\textit{non-live migration}\xspace}
\newcommand{\trustzone}{\textit{TrustZone}\xspace}
\newcommand{\docker}{\textit{Docker}\xspace}
\newcommand{\critical}{\textit{critical}\xspace}
\newcommand{\high}{\textit{high}\xspace}
\newcommand{\moderate}{\textit{moderate}\xspace}
\newcommand{\low}{\textit{low}\xspace}
\newcommand{\boot}{\textit{boot}\xspace}
\newcommand{\handler}{\textit{handler}\xspace}
\newcommand{\bytes}{\textit{bytes}\xspace}
\newcommand{\frames}{\textit{frames}\xspace}
\newcommand{\Frames}{\textit{Frames}\xspace}
\newcommand{\Bitmap}{\textit{Bitmap}\xspace}
\newcommand{\bitmap}{\textit{bitmap}\xspace}
\newcommand{\myloop}{\textit{loop}\xspace}
\newcommand{\myloops}{\textit{loops}\xspace}
\newcommand{\default}{\textit{default}\xspace}
\newcommand{\setup}{\textit{setup}\xspace}
\newcommand{\namespace}{\textit{namespace}\xspace}
\newcommand{\Namespace}{\textit{Namespace}\xspace}
\newcommand{\Namespaces}{\textit{Namespaces}\xspace}
\newcommand{\namespaces}{\textit{namespaces}\xspace}
\newcommand{\mount}{\textit{mount}\xspace}
\newcommand{\hostname}{\textit{hostname}\xspace}
\newcommand{\conteinerd}{\textit{Conteinerd}\xspace}
\newcommand{\mac}{\textit{macOS}\xspace}
\newcommand{\windows}{\textit{Windows}\xspace}


\newcommand{\kmigrateto}{\texttt{kmigrate\_to}\xspace}

\newcommand{\showtodos}{1}

\if\showtodos1
    \newcommand{\mytodo}[1]{\todo[inline,color=red!30]{#1}}
\else
    \newcommand{\mytodo}[1]{}
\fi

\setlist[enumerate,1]{%
  label=\arabic*.,
}

\newlist{inlinelist}{enumerate*}{1}
\setlist*[inlinelist,1]{%
  label=(\roman*),
}
\addtocontents{toc}{\protect\setcounter{tocdepth}{2}}

\begin{document}

    \pretextual%
    \imprimircapa%
\imprimirfolhaderosto*
% Atenção! esse \protect é importante
%\protect\incluirfichacatalografica{ficha.pdf}
\imprimirfolhadecertificacao

%\begin{dedicatoria}
  Dedico este trabalho aos meus pais,
  
  aos demais membros da família e aos meus amigos.
\end{dedicatoria}
%\begin{agradecimentos}
  Agradeço a Deus pela minha vida.
  
  Agradeço aos meus pais e minha família, que sempre se me motivaram e confiaram na minha capacidade de superar os obstáculos da vida.
  
  Agradeço a todos que de alguma forma contribuiram no desenvolvimento deste trabalho e me auxiliaram na jornada de me tornar um profissional mais capaz. Em especial, agradeço ao meu professor orientador Márcio Bastos Castro, ao meu coorientador João Vicente Souto e aos demais colegas de projeto. 
  
  Agradeço ao Conselho Nacional de Desenvolvimento Científico e Tecnológico pelo auxílio através do Programa Institucional de Bolsas de Iniciação Cientifica (PIBIC).
  
  Agradeço aos meus amigos de curso pela convivência intensa e companheirismo durante os últimos anos.
  
\end{agradecimentos}
%\begin{epigrafe}
    So we keep asking, over and over, until a handful \\
    of earth stops our mouths — but is that an answer?

    (Heine, H., The Lazarus Poems, 1851)
\end{epigrafe}
\begin{resumo}[Resumo]
  A classe de processadores \lw surgiu para prover um alto grau de paralelismo e eficiência energética. Contudo, o desenvolvimento de aplicações para esses processadores enfrenta diversos problemas de programabilidade provenientes de suas peculiaridades arquitetônicas. Especialmente, o gerenciamento de processos precisa mitigar problemas provenientes das pequenas memórias locais e da falta de um suporte robusto para virtualização. Nesse contexto, este trabalho visa desenvolver a funcionalidade de migração de processos em um \os distribuído para \lws através de uma abordagem de virtualização leve baseada em contêineres. Particularmente, este trabalho está incluído no projeto \nanvix, um \os distribuído baseado em uma abordagem \multikernel de código aberto projetado para \lws. Os resultados experimentais mostram que a virtualização impactou positivamente o desempenho do \so. Houve aumento de desempenho no subsistema de \threads e redução de desvios, faltas na \dcache e faltas na \icache. Os processos puderam ser tranferidos entre \clusters do processador em um \downtime que varia entre 19~ms e 101~ms, dependendo da quantidade de recursos utilizados.

  % Atenção! a BU exige separação através de ponto (.). Ela recomanda de 3 a 5 keywords
  \vspace{\baselineskip} 
  \textbf{Palavras-chave:} Lightweight Manycores. Sistemas Operacionais. Migração de Processos. Virtualização. Conteinerização
\end{resumo}

\begin{abstract}
  
The lightweight manycore processor class emerged to provide a high degree of parallelism and energy efficiency. However, developing applications for these processors faces various programmability issues stemming from their architectural peculiarities. Particularly, process management needs to mitigate problems arising from small local memories and the lack of robust virtualization support. In this context, this work aims to develop a process migration functionality in a distributed operating system for lightweight manycores through a lightweight container based virtualization approach. Specifically, this work is part of the Nanvix project, which is an open-source  distributed operating system based on a multikernel approach designed for lightweight manycores. Experimental results show that virtualization positively impacted the operating system's performance. There was an increase in performance in the thread subsystem and a reduction in branches, in data cache misses and instruction cache misses. The processes were able to be transferred between processor clusters with a down time ranging from 19~ms to 101~ms, depending on the amount of resources used.

  \vspace{\baselineskip} 
  \textbf{Keywords:} Lightweight Manycores. Operating Systems. Process Migration. Virtualization. Containerization
\end{abstract}

\listoffigures*

\listadesiglas[9em]
% Lista para ambiente algorithm
% \listofalgorithms*

% \begin{listadesimbolos}
%   $\gets$   & Atribuição \\
%   $\exists$   & Quantificação existencial \\
%   $\rightarrow$   & Implicação \\
%   $\wedge$   & E lógico \\
%   $\vee$   & Ou lógico \\
%   $\neg$   & Negação lógica \\
%   $\mapsto$   & Mapeia para \\
%   $\sqsubseteq$   & Subclasse (em ontologias) \\
%   $\subseteq$   & Subconjunto: $\forall x\;.\; x \in A \rightarrow x \in B$ \\
%   $\langle\ldots\rangle$ & Tupla \\
%   $\forall$   & Quantificação universal \\
%   mmmmm & Nenhum sentido, apenas estou aqui para demonstrar a largura máxima dessas colunas. Ao abrir o ambiente \texttt{listadesimbolos}, pode-se fornecer um argumento opcional indicando a largura da coluna da esquerda (o default é de 5em): \texttt{\textbackslash{}begin\{listadesimbolos\}[2cm] .... \textbackslash{}end\{listadesimbolos\}} \\
% \end{listadesimbolos}

\tableofcontents*%
    
    %%%%%%%%%%%%%%%%%%%%%%%%%%%%%%%%%%%%%%%%%%%%%%%%%%%%%%%%%%%%%%%%%%%%
    %%% Corpo do texto                                               %%%
    %%%%%%%%%%%%%%%%%%%%%%%%%%%%%%%%%%%%%%%%%%%%%%%%%%%%%%%%%%%%%%%%%%%%
    \textual%
    \chapter{Introdução}
\label{chap.intro}
Durante anos o aumento do desempenho em processadores esteve associada ao aumento da frequência interna nos processadores e avanços na tecnologia dos semicondutores. Essas técnicas se manteveram eficientes até o momento em que a dissipação de calor interna dos \chips inviabilizou o aumento da frequência dos processadores. Isso, associado ao fim da lei de Moore \cite{moore:1965} fez com que novas maneiras de se aumentar o poder computacional fossem exploradas.

Como alternativa para o aumento de desempenho, foram desenvolvidos 
os processadores com vários núcleos de processamento, os \multicores,
cujo desempenho vem aliado também à quantidade de núcleos, e não mais 
apenas às altas frequências de relógio. Desse modo, mesmo com a estabilização da frequência nos processadores, esse aumento na quantidade de \cores em conjunto com outras melhorias no \hardware, como o aumento no número de transistores nos \chips, aperfeiçoamento dos preditores de desvio e adaptações na hierarquia de memória, o desempenho dos sistemas computacionais continuaram a ampliar.

Atualmente, a eficiência energética dos sistemas computacionais
revela-se tão importante quanto o desempenho. Segundo o Departamento de \darpa, a potência recomendada para um supercomputador atingir o \exascale
($10^{18}$ \flops), é de 20 MW, o que é inviável para a realidade dos sistemas computacionais modernos \cite{darpa:exascale}. Nesse cenário surge a classe dos processadores \lws. Esses processadores são classificados como \mpsocs e tem como objetivo justamente atrelar alto desempenho à eficiência energética \cite{francesquini2015}. Para atingir esse objetivo, a arquitetura dessa classe de processadores é caracterizada por:
\begin{enumerate}[label=(\roman*)]
    \item Integrar centenas ou milhares de núcleos de processamento operando a baixas frequências em um único chip;
    \item Operar sobre \mimd;
    \item Organizar os núcleos em conjuntos, denominados \clusters, para compartilhamento de recursos locais;
    \item Utilizar \nocs para transferência de dados entre núcleos ou \clusters;
    \item Possuir sistemas de memória distribuídos e restritivos; e
    \item Apresentar componentes heterogêneos.
\end{enumerate}
Os processadores \mppa \cite{dinechin:2013}, PULP \cite{pulp} e \taihulight \cite{fu2016sunway} são exemplos comerciais dessa classe de processadores. Uma visão conceitual da arquitetura de um \lw é ilustrada pela Figura \ref{fig.lw-overview}.

\begin{figure}[bt]
	\label{fig.lw-overview}
	\centering
	\includegraphics[width=0.5\linewidth]{content/images/lw-overview-gs.jpg}
	\caption{Visão conceitual de um processador \lw \cite{penna2021inter}}
\end{figure}


Apesar de os processadores \lws serem uma alternativa às abordagens tradicionais no que se refere ao aumento de desempenho, as características arquiteturais ainda induzem problemas de programabilidade nas aplicações paralelas \cite{Castro-PARCO:2016}. Entre eles podem-se citar:

\begin{enumerate}[label= (\roman*)]
    \item Modelo de programação híbrida que força troca de informação entre os \clusters exclusivamente por troca de mensagens via \noc \cite{kelly2013};
    \item Sistema de memória restritivo, em que há multiplos espaços de endereçamento, pequena memória local, necessidade de busca em memória remote e separação da memória em pequenos blocos explicitament para a manipulação dos dados \cite{Castro-PARCO:2016};
    \item Latência e gargalos de comunicação na \noc;
    \item Falta de suporte de coerência de cache para economia de energia, o que exige do programador a gerência de cache via \software;
    \item Configuração heterogênea no que se refere aos \cclusters e \ioclusters, o que dificulta o desenvolvimento de aplicações;
\end{enumerate}

Atualmente, alguns estudos são feitos para amenizar o impacto da arquitetura sobre o desenvolvimento de aplicações. Neles, sobresaem-se os \oss distribuídos, que garantem um ambiente mais robusto e rico \cite{asmussen_m3:_2016, kluge_operating_2014, penna:sbesc19}. Destaca-se ainda os estudos em \oss ditribuídos baseados em uma abordagem \multikernel \cite{penna2017-1,penna2017-2,penna2019}.

Nesse cenário, a virtualização dos recursos do processador é importante para o suporte a multi-aplicação e para maior eficiência do mesmo \cite{vanz2022virtualizaccao}. Contudo, as características arquiteturais dos \lws, especialmente relacionadas à memória, inviabilizam um suporte complexo para virtualização. Por exemplo, máquinas virtuais utilizadas em ambientes \cloud possuem à disposição centenas de GBs para isolar duplicatas inteiras de \oss com a ajuda de virtualização no nível de instrução \cite{sharma2016containers}. Nos \lws, as pequenas memórias locais e a simplificação do \hardware para reduzir o consumo energético restringem os tipos de virtualização suportados.

Neste contexto, este trabalho trabalho explora um modelo mais leve de virtualização para \lw baseada em contêineres. Contêineres são executados pelo \os como aplicações virtuais e não incluem um \os convidado, resultando em um menor impacto no sistema de memória e requerendo menor complexidade do \hardware \cite{thalheim2018cntr, sharma2016containers}.

\glsresetall
\section{Objetivos}
\label{sec.goals}

Com base nas motivações citadas previamente. Os objetivos deste trabalho serão especificados nas próximas seções.

\subsection{Objetivo Principal}
\label{sec.goals.primary}

O objetivo principal deste trabalho é adaptar o \nanvix, um \os para \lws, de modo que os recursos utilizados por um processo sejam virtualizados. Isso com o objetivo de desvincular a execução de um processo com o local \ie \cluster onde está alocado e aumentar a mobilidade de processos no processador.

\subsection{Objetivos Específicos}
\label{sec.goals.secondary}

\begin{enumerate}[label= (\roman*)]
    \item Propor um modelo de virtualização adaptado às necessidades e imposições de um \lw;
    \item Implementar o modelo proposto no \nanvix, um \so distribuído para \lws;
    \item Analisar a corretude da solução através do desenvolvimento de \benchmarks que avaliem a migração de processos;
    \item Analisar o impacto do modelo de virtualização na execução normal do \nanvix;
\end{enumerate}

\section{Organização do Trabalho}
\label{sec.organization}

Os próximos capítulos do trabalho estão organizadas da seguinte maneira. No \autoref{chap.background} serão apresentados alguns conceitos importantes para o melhor entendimento do trabalho. Dentre esses conceitos pode-se citar:
\begin{inlinelist}[label= (\roman*)]
    \item \Lws;
    \item Multiprocessadores;
    \item Multicomputadores;
    \item Virtualização.
\end{inlinelist}
Além disso, será detalhado o \os e o \lw que será utilizado neste trabalho. No \autoref{chap.related-work} serão apresentados alguns trabalhos relacionados a este, bem como serão destacados as semelhanças e diferenças desse trabalho com os apresentados. No \autoref{chap.dev.virtualizacao} serão expostos a proposta e os detalhes do desenvolvimento da solução. No \autoref{chap.results} serão elucidados os resultados dos testes avaliadores da solução. Por fim, no \autoref{chap.conclusions} serão exibidas as conclusões obtidas.

\glsresetall
\chapter{Referencial Teórico}
\label{chap.background}

Neste capítulo serão apresentados alguns conceitos importantes para o entendimento do trabalho. Na Seção \ref{sec.lw} será apresentada uma visão geral de como os processadores evoluíram de \singlecores aos \lws. Na Seção \ref{sec.nanvixos} será apresentado o \so que será utilizado no desenvolvimento deste trabalho, o \nanvix. Na Seção \ref{sec.virtualizacao} será explicado um pouco sobre a virtualização e migração de processos, que é o tema principal do trabalho.

\section{Dos \singlecores aos \Lws}
\label{sec.lw}

Durante anos o aumento do desempenho dos processadores se manteve uma necessidade constante para o avanço da ciência em vários setores: astrologia, biologia, engenharia, etc. Até tempos atrás esse objetivo era alcançado através do aumento da frequência interna de \singlecores, do avanço na tecnologia dos semicondutores e do acréscimo do número de transistores em um \chip. Atualmente, estamos chegando a um limite físico que impede a aplicação dessas técnicas. Além da dificuldade de garantir o controle da dissipação de calor à medida que a frequência aumenta, o número de transistores que conseguimos colocar em um \chip está se estabilizando, haja vista o aparente impedimento na diminuição significativa do tamanho de um transistor.

Como alternativa para a continuidade nos avanços de poder computacional, foram exploradas novas técnicas. Em especial, foram desenvolvidas as arquiteturas paralelas, que exploram o poder de processamento paralelo, o qual é atingido pela execução de múltiplos \cores simultaneamtne. Essas novas arquiteturas são classificadas de acordo com a maneira com que conseguem manipular os dados. São elas:
\begin{inlinelist}
    \item \sisd;
    \item \simd;
    \item \misd;
    \item \mimd.
\end{inlinelist}
Neste trabalho, as mais relevantes são as arquiteturas que suportam cargas de trabalho \mimd, as quais ainda podem ser divididas em multiprocessadores ou multicomputadores, como mostrado na Figura \ref{fig.mimd} \cite{tanenbaum:4ed}. 

No presente trabalho, destaca-se a classe de processadores \lw, que pode ser classificada como \mpsoc. \Lws tem como objetivo atrelar o alto poder de processamento paralelo com eficiência energética. Para isso sua arquitetura segue as seguintes características:
\begin{enumerate}[label=(\roman*)]
    \item Integrar centenas ou milhares de núcleos de processamento operando a baixas frequências em um único chip;
    \item Operar sobre \mimd;
    \item Organizar os núcleos em conjuntos, denominados \clusters, para compartilhamento de recursos locais;
    \item Utilizar \nocs para transferência de dados entre núcleos ou \clusters;
    \item Possuir sistemas de memória distribuídos e restritivos, com pequenas memórias locais;
    \item Apresentar componentes heterogêneos (\cclusters e \ioclusters).
\end{enumerate}


\begin{figure}[bt]
    \centering
    \includesvg[width=0.9\linewidth]{content/images/mimd.svg}
    \caption{(a) um multiprocessador de memória compartilhada. (b) um multicomputator com troca de mensagens. (c) um sistema distribuído de grande escala.\cite{tanenbaum:4ed}}\label{fig.mimd}
\end{figure}

Alguns exemplos comerciais bem sucedidos de \lws são o \mppa \cite{dinechin:2013}, PULP \cite{pulp} e \taihulight \cite{fu2016sunway}. Uma visão conceitual da arquitetura de um \lw é ilustrada pela Figura \ref{fig.lw-overview}.

Mais detalhadamente, para o desenvolvimento deste trabalho será utilizado o processador \mppa. A Figura \ref{fig.arch-mppa} apresenta uma visão geral do processador e suas peculiaridades, tais como:

\begin{enumerate}[label=(\roman*)]
    \item integrar 288 núcleos de baixa frequência em um único chip;
    \item organizar os núcleos em 20 conjuntos (\clusters) para compartilhamento de recursos locais;
    \item utilizar 2 \nocs para transferência de dados entre \clusters;
    \item possuir um sistema de memória distribuída composto por pequenas memórias locais, \eg  \sram de 2 MB;
    \item não dispor de coerência de \cache;
    \item apresentar componentes heterogêneos, \eg \clusters destinados à computação ou comunicação com periféricos (\cclusters e \ioclusters, respectivamente).
\end{enumerate}

\begin{figure}[bt]
    \centering
    \includegraphics[width=0.6\linewidth]{content/images/arch-mppa-gs.png}
    \caption{Visão arquitetural do processador \mppa \cite{penna:sbesc19}}\label{fig.arch-mppa}
\end{figure}

\section{\nanvixos}\label{sec.nanvixos}

O \nanvix\footnote{Disponível em https://github.com/nanvix} é um \os distribuído e de propósito geral que busca equilibrar desempenho, portabilidade e programabilidade para \lws \cite{penna:sbesc19}. O \nanvix é estruturado em 3 camadas de \kernel. São elas:
\begin{description}
    \item [\nanvix \hal]
         é a camada mais baixa que abstrai e provê o gerenciamento dos recursos de \hardware sobre uma visão comum \cite{penna:hal}. Entre esses recursos estão: \cores, \tlbs, \cache, \mmu, \noc, interrupções, memória virtual, recursos de \io. De maneira geral, esta camada provê visões a nivel de \core, \cluster e comunicação/sincronização entre \clusters \cite{penna:thesis}. A Figura \ref{fig.hal-overview} ilustra a estrutura interna da \hal do \nanvix.
    \item [\nanvix \Assymetric \Microkernel]
        é a camada intermediária que provê gerenciamento de recursos e os serviços mínimos de um \os em um \cluster. Entre esses serviços se encontram a comunição intercluster, gerenciamento de \threads e memória, controle de acesso à memória e interface para chamadas de sistema. As chamadas de sistema podem ser executadas localmente, caso acessem dados \rdo ou alterem estruturas internas do \core, ou remotamente pelo \mcore, que atende à requisição e libera o \score requisitante ao seu término \cite{penna:thesis}. Essa característica adjetiva o \microkernel como assimétrico. A Figura \ref{fig.microkernel-overview} ilustra a estrutura interna do \microkernel do \nanvix.    
    \item [\nanvix \Multikernel]
        é a camada superior que provê os serviços de um \os e dispõe uma visão a nível do processador em si. Os serviços são hospedados em \clusters \ie isolados das aplicações de usuário e atendem as requisições vindas dos processos de usuário através de um modelo cliente-servidor. As requisições e respostas são enviadas/recebidas através de passagem de mensagem via \noc. Os serviços dessa camada podem ser entendidos como fontes de informação que mantém a execução dos processos consistentes no processador, tendo em vista a natureza distribuída da memória nessas arquiteturas. Nesses serviços estão incluídos mecanismos de \spawn de processos e obtenção de nomes lógicos dos processos (a fim de localizá-los para comunicação), por exemplo.
\end{description}

\begin{figure}[bt]
    \centering
    \includesvg[width=0.8\linewidth]{content/images/hal.svg}
    \caption{Estrutura interna da \hal do \nanvix \cite{penna:thesis}}\label{fig.hal-overview}
\end{figure}

\begin{figure}[bt]
    \centering
    \includesvg[width=0.8\linewidth]{content/images/microkernel.svg}
    \caption{Estrutura interna do \microkernel do \nanvix \cite{penna:thesis}}\label{fig.microkernel-overview}
\end{figure}

Em sua abordagem original, os processos no \nanvix são estáticos, \ie cada \cluster possui apenas um processo. Desse modo, uma vez que o processo inicia sua execução em um \cluster, este finalizará a execução no mesmo \cluster. 
Isso torna o processo dependente do \cluster que o executa \eg a comunicação entre processos está atrelada aos \clusters nos quais os processos são executados e não aos processos em si. A falta de mobilidade dos processos nesse modelo pode trazer sobrecargas ao processador e afeta o suporte a multi-aplicação. Por exemplo, a comunicação entre \clusters próximos é mais rápida e resulta em menor consumo energético do processador. Sendo assim, melhorar a mobilidade e a disposição dos processos no processador \ie viabilizar a migração de processos entre \clusters, possibilitaria melhorar o gerenciamento dos recursos do mesmo. Desse modo, este trabalho explora justamente essa dissassociação do \hardware com a execução do processo. Isso com o objetivo de desvincular o processo do \cluster que o executa, o que, por fim, aumentaria a mobilidade dos processos \ie a migração deles entre os \clusters.

\subsection{Abstrações de Comunicação do \nanvix}
O \nanvix dispõe de três abstrações de comunicações para transferência de dados e sincronização entre \clusters \cite{penna:thesis}. Nas próximas Seções serão detalhadas as três abstrações.

\begin{figure}[tb]
	\centering
	\subcaptionminipage[fig.sync1n]%
                   {.44\textwidth}
                   {Modo $1:N$.}
                   {\includesvg[width=\textwidth]{content/images/sync-1-n.svg}}
	\qquad
	\subcaptionminipage[fig.syncn1]
                   {.44\textwidth}
                   {Modo $N:1$.}
                   {\includesvg[width=\textwidth]{content/images/sync-n-1.svg}}
	\caption{Fluxo de execução da abstração \sync \cite{penna:thesis}.\label{fig.sync}}
\end{figure}

\subsubsection{\sync}
Esta abstração é a que da o suporte a sincronização inter-kernel. Através dela um processo pode esperar um sinal, que pode ser disparado por outro processo remotamente através das interfaces \noc. Essa abstração é muito utilizada na inicialização do sistema para garantia de sincronização entre os subsistemas \cite{penna:thesis}.

O \sync pode ser operado duas maneiras distintas: o modo $1:N$ e $N:1$. No modo $1:N$ (Figura \ref{fig.sync1n}) um nó envia uma notificação a múltiplos nós, que estão esperando pelo sinal. Em contraste, no moso $N:1$ (Figura \ref{fig.syncn1}), múltiplos nós enviam uma notificação a um único nó \cite{penna:thesis}.

\begin{figure}[bt]
    \centering
    \includesvg[width=0.5\linewidth]{content/images/mailbox.svg}
    \caption{Fluxo de execução da abstração \mailbox \cite{penna:thesis}}\label{fig.mailbox}
\end{figure}

\subsubsection{\mailbox}
Esta abstração é responsável pelo suporte à troca de mensagens de controle. Isso através de troca assíncrona de pequenas mensagens de tamanho fixo. A abstração segue a semântica $N:1$ e o funcionamento é o seguinte: um nó (destinatário da mensagem) possuí um \mailbox, do qual lê mensagens, e múltiplos nós (remetentes da mensagem) podem escrever nesse \mailbox \cite{penna:thesis}. A Figura \ref{fig.mailbox} ilustra o fluxo de execução da \mailbox.

\begin{figure}[b]
    \centering
    \includesvg[width=0.5\linewidth]{content/images/portal.svg}
    \caption{Fluxo de execução da abstração \portal \cite{penna:thesis}}\label{fig.portal}
\end{figure}

\subsubsection{\portal}
Esta abstração é reponsável pela troca de largas mensagens e segue a semântica $1:1$. A abstração pode ter uso em diversos cenários que exigem grandes transferências de dados entre \clusters \cite{penna:thesis}. A Figura \ref{fig.portal} ilustra o fluxo de execução da abstração \portal.
    

\section{Virtualização e Migração de Processos}
\label{sec.virtualizacao}

A virtualização pode ser entendida como a desvinculação da execução de uma aplicação (ou até um \so) dos recursos físicos responsáveis pelo seu funcionamento. O desacoplamento entre aplicação e \hardware, quando há virtualização, pode permitir, dependendo da camada em que esta é aplicada, a existência simultânea e isolada de múltiplas instâncias de usuários ou \oss (\vms), que compartilham e concorrem pelos mesmos recursos de \hardware reais. Reaproveitamento de recursos, portabilidade e segurança são algumas vantagens proporcionadas pela virtualização.

Em ambientes \cloud é muito comum a utilização de \vms para execução de tarefas nos servidores. Com o auxílio da virtualização, um único servidor pode alocar diversas \vms, possivelmente com \sos distintos. 

Neste trabalho, o foco é a virtualização de processos. Isto é, o objetivo é desacoplar a execução de uma aplicação do \cluster do \lw que a executa. Na abordagem original do \nanvix, o processo é dependente do local em que é alocado, o que afeta o suporte a migração e diminui a eficiência computacional, como detalhado na Seção \ref{sec.nanvixos}. Nesse contexto, a virtualização é útil justamente para aumentar a mobilidade dos processos, o que possibilitaria o gerenciamento da distribuição dos processos no processador. Particularmente, este trabalho trabalho explora um modelo mais leve de virtualização para \lw baseada em contêineres. Contêineres são executados pelo \os como aplicações virtuais e não incluem um \os convidado, resultando em um menor impacto no sistema de memória e requerendo menor complexidade do \hardware \cite{thalheim2018cntr, sharma2016containers}.
\glsresetall
\chapter{Trabalhos Relacionados}
\label{chap.related-work}

Neste capítulo, serão mostradas técnicas e pesquisas que estão sendo desenvolvidas no que diz respeito à virtualização e migração. Serão apresentados trabalhos relacionados, bem como serão evidenciadas as semelhanças e diferenças com o presente trabalho.

Grande parte das pesquisas relacionadas à migração estão inseridas em ambientes \cloud. Nesses casos, os esforços estão voltados para redução do tempo total de migração, diminuição do \downtime~\cite{migration-linux-conteiners,clark2005live} e exploração/otimização das vantagens que a migração de processos oferece nesses ambientes computacionais. Entre essas vantagens podem-se citar:
\begin{enumerate}[label=(\roman*)]
    \item Balanceamento de carga~\cite{live-vm-migration-techniques,ada-things};
    \item Tolerância a falhas~\cite{fernando2019live};
    \item Gerenciamento do consumo de energia~\cite{aldossary2018performance};
    \item Compartilhamento de recursos; e
    \item Manutenção de sistemas sem interrupções~\cite{live-vm-migration-techniques,ada-things}.
\end{enumerate}

Apesar da maioria das pesquisas estarem voltadas à exploração desses benefícios e diminuição do tempo de migração e \downtime em ambientes \cloud, há alguns autores preocupados com o desenvolvimento de soluções envolvendo virtualização e migração em ambientes de recursos restritos, como sistemas de tempo real e sistemas críticos, nos quais há restrições de tempo também. Dessa forma, como a temática de limitação de recursos, especialmente de memória, é muito presente neste trabalho, serão abordados nas próximas seções algumas pesquisas desses autores, \ie pesquisas voltadas à busca pelo uso da virtualização/migração de forma mais leve e cujo impacto no \hardware seja reduzido, adaptando-se a esses sistemas de recursos limitados.

\section{\textit{''Virtualization on TrustZone-enabled Microcontrollers?\\ Voilà!''}}\label{sec.rw-1}

O artigo \textit{''Virtualization on TrustZone-enabled Microcontrollers? Voilà!''}~\cite{pinto2019virtualization} aborda a possibilidade de implementação da virtualização em microcontroladores que utilizam \trustzone. \trustzone é uma tecnologia de \hardware voltada à segurança, em que a execução de um sistema pode ser dividida entre normal e segura. Os autores afirmam que essa tecnologia pode ser explorada além das suas propriedades de segurança. Isso porque o \trustzone também provê certo nível de isolamento dos recursos, o que o torna viável de ser usado para virtualização, afinal o isolamento cria um ambiente seguro e propício para a execução simultânea e isolada de múltiplas \vms.

\citeonline{pinto2019virtualization} expõe a dificuldade de se implementar a virtualização em \mcus devido aos seus recursos limitados. Nesses ambientes, não é possível a utilização de \hypervisors tradicionais, haja vista a baixa complexidade de \hardware das \mcus. Sendo assim, para atender a necessidade de baixo impacto nos recursos dos \mcus, os autores propõem uma solução que usa um \hypervisor mais leve para gerenciar as \vms nesses ambientes utilizando a tecnologia \trustzone para garantir o isolamento das \vms.

Os testes foram feitos num microcontrolador \textit{Cortex-M4} e a solução proposta garante o suporte à execução múltipla de \vms em microcontroladores.

\section{\textit{''Checkpointing and migration of IoT edge functions''}}\label{sec.rw-2}


O artigo \textit{''Checkpointing and migration of IoT edge functions''}~\cite{karhula2019checkpointing} propõe um artifício envolvendo migração de contêineres entre dispositivos \iot de borda como solução para a diminuição do uso de recursos em dispositivos \iot.

Os autores evidenciam que os aparelhos \iot são usados na computação de borda para promover o que chamamos de \faas, que é um tipo de serviço oferecido por diversas plataformas, como a \textit{Amazon AWS Lambda} e \textit{Google Cloud Functions}. Contudo, dispositivos \iot possuem recursos limitados, restringindo-se à execução de poucos contêineres simultaneamente. Além disso, as abordagens tradicionais de \faas sugerem a execução ininterrupta dos contêineres que são iniciados. Isso torna a computação de borda ineficiente, pois esse esquema pode sobrecarregar rapidamente os dispositivos \iot, haja vista a memória limitada desses. A situação se agrava ainda mais quando consideramos funções de longa duração bloqueantes (muito comuns em sistemas de autenticação) \eg funções que esperam alguma requisição, resposta ou qualquer tipo de sinal de outro sistema, seja ele um outro dispositivo \iot ou uma ação humana.

Dessa forma, \citeonline{karhula2019checkpointing} propõem um esquema de \checkpointing utilizando \docker e \criu. Através dessas tecnologias, os contêineres que não estão executando computação útil são interrompidos e salvos em disco, liberando espaço da memória para a execução de outro contêiner. Isso se torna extremamente útil quando consideramos funções de longa duração bloqueantes, já que durante o tempo de espera pelo sinal, a aplicação pode ser interrompida. Além disso, com o estado salvo em disco, a migração de contêineres entre dispositivos \iot de borda se torna possível. Dessa forma, além de reduzir o uso de recursos nos dispositivos de computação em borda, através da migração dos contêineres, outros benefícios surgem, como o balanceamento de carga e tolerância a falhas entre aparelhos \iot de borda.

Os testes foram feitos em uma \textit{Raspberry Pi 2 Model B}, a qual rodava diversos contêineres com aplicações em \textit{Node JS} de longa duração e que simulavam o comportamento bloqueante. Os resultados apontam que houve economia no uso de recursos, em especial da memória, e que a migração de contêineres entre dispositivos \iot de borda é possível.

\section{ \textit{''Lightweight virtualization as enabling technology for future smart cars''}}\label{sec.rw-3}

O artigo \textit{''Lightweight virtualization as enabling technology for future smart cars''} \cite{smartcarslwvirtualization} discorre sobre a possibilidade de usar a virtualização no desenvolvimento de aplicações para carros inteligentes. Os sistemas presentes nos carros inteligentes também têm certa limitação de recursos que dificultam a aplicação direta de \hypervisors tradicionais, muito comuns em ambientes \cloud.

Sendo assim, os autores propõem um sistema que utiliza contêineres \docker para criar uma camada de abstração a nível de processo. Dessa forma, cada aplicação é executada em um contêiner distinto. Esse sistema tem impacto menor nos recursos de \hardware e é suficiente para garantir a execução isolada das aplicações virtuais (contêineres).

Além disso, o sistema engloba um escalonador de contêineres, que é responsável por gerenciar os contêineres e o \hardware alocado para cada um. Ademais, tem finalidade de sinalizar a instanciação e destruição dos contêineres conforme a necessidade. Esse escalonador é capaz de gerenciar os recursos de \hardware de forma a garantir que os contêineres sejam executados de maneira eficiente, sem que haja desperdício de recursos. No modelo proposto pelos autores, há 4 tipos de tarefas: \critical, \high, \moderate e \low. Cada um desses tipos possui um nível de prioridade, sendo que o \critical é o mais prioritário e o \low é o menos prioritário. O escalonador é responsável por garantir que as tarefas de maior prioridade sejam executadas primeiro. Tarefas relacionadas à segurança dos passageiros \eg sistemas de alerta ou câmera são consideradas mais prioritárias que tarefas relacionadas à sistemas de entretenimento \eg sistemas de áudio ou vídeo.

A proposta foi testada em uma \textit{Raspberry Pi 3} e os resultados foram considerados positivos. Os contêineres garantiram a execução do sistema de maneira a considerar a limitação de \hardware e suportaram a execução paralela de múltiplas aplicações. O escalonador de contêineres foi capaz de gerenciar os recursos de maneira eficiente, priorizando as tarefas de maior prioridade.

\section{\textit{''Container-based real-time scheduling in the linux kernel''}}\label{sec.rw-4}

O artigo \textit{''Container-based real-time scheduling in the linux kernel''} \cite{abeni2019container} aborda o tema de virtualização com contêineres em sistemas de tempo real. Os autores exploram a implementação de um escalonador de tarefas em sistemas de tempo real utilizando \lxc (\ie com o auxílio de \cgroups e \namespaces). 

\citeonline{abeni2019container} explicam que os componentes que constituem os ambientes de sistemas de tempo real tradicionalmente eram executados por uma \vm dedicada, em que as \vms eram gerenciadas por um \hypervisor escalonador. Os autores afirmam que em algumas situações (\eg em sistemas embarcados) pode ser vantajosa a utilização de uma virtualização mais leve, baseada em contêineres, já que este tipo de virtualização sobrecarrega menos o sistema quando comparada com a abordagem de virtualização total tradicionalmente utilizada.

Desse modo os autores propõem um escalonador que estende um escalonador já presente no \kernel do \linux, o SCHED\_DEADLINE. A implementação consiste em um sistema de escalonamento em uma hierarquia de dois níveis. No primeiro nível, o qual é responsável pelo escalonamento dos \cgroups, é utilizada a política de escalonamento do SCHED\_DEADLINE e no segundo nível é utilizada uma política de escalonamento com prioridade fixa, como o SCHED\_FIFO ou SCHED\_RR.
Resumidamente, no momento em que a política de escalonamento SCHED\_DEADLINE escalona uma entidade que está associada a uma fila de \tasks de tempo real, então o escalonador de prioridade fixa seleciona a \task de maior prioridade.

Os experimentos mostraram que o escalonador proposto fez com que os \cores fossem utilizados de um forma melhor, quando compara-se com escalonadores que utilizam a virtualização total. Isso porque, em contraste com a virtualização total, na virtualização com contêineres, os componentes compartilham o mesmo \kernel. Sendo assim, é possível a migração de \tasks para filas de \cores menos utilizados.

\section{Comparação do Presente Trabalho com os Trabalhos Relacionados}

O presente trabalho se assemelha com os trabalhos relacionados apresentados no sentido de aplicar a virtualização e migração em um sistema com recursos restritos. De maneira similar aos trabalhos detalhados nas Seções \ref{sec.rw-2}, \ref{sec.rw-3} e \ref{sec.rw-4} e em contraste com o trabalho detalhado na Seção \ref{sec.rw-1}, neste trabalho foi explorado um modelo de virtualização baseado na utilização de contêineres. Isso foi feito com o intuito de evitar a sobrecarga que a virtualização total tem sobre o ambiente, em especial na memória, que nos \lws é escassa.

Neste trabalho, não foi utilizada qualquer ferramenta de auxílio na manipulação de contêineres, em contraste com os trabalhos apresentados, que usufruem de mecanismos de gerenciamento de contêineres, como o Docker~\cite{smartcarslwvirtualization,karhula2019checkpointing} ou o \lxc~\cite{abeni2019container}. Isso porque o trabalho é feito sobre um ambiente não usual, os \lws, utilizando um \so ainda incomum para a prática dessas técnicas, o \nanvix. Dessa forma, a criação de um contêiner e a forma como este é manipulado foram desenvolvidas desde o início e especificamente para o \nanvix. Isso traz à implementação algumas características peculiares. Por exemplo, os contêineres executam sobre uma estrutura de \kernel idêntica, se diferenciando pelo contexto do processo, o que é uma característica comumente vista na virtualização a nível de \so. Contudo, o \kernel não é compartilhado entre os contêineres, já que em um \cluster temos exatamente um contêiner e os \clusters são independentes entre si. Logo, um contêiner tem acesso completo aos recursos do \cluster em que executa e o \cluster tem visão apenas do contêiner que hospeda. Então, o gerenciamento desses contêineres precisa ser feito em um nível de abstração maior, em que se tem uma visão do processador como um todo, e não somente de um \cluster.

Além disso, em contraste com os trabalhos apresentados, a principal vantagem explorada com a virtualização é o aumento da mobilidade dos processos, possibilitando a migração de processos. A utilização eficiente dos recursos promovida pela virtualização, mesmo que necessária nos \lws pela limitação de recursos computacionais (em especial a memória) se torna uma vantagem indireta da virtualização. Isso porque o uso eficiente de \hardware é provido mais pela migração (através da melhor disposição dos processos entre os \clusters) do que pela virtualização em si.


\glsresetall
\chapter{Proposta de Virtualização e Migração de Processos para \textit{Lightweight Manycores}}
\label{chap.dev.virtualizacao}

Este trabalho de conclusão propõe-se a aumentar a independência dos processos no processador através do projeto e desenvolvimento do suporte à virtualização e migração de processos em \lws. Ambientes \cloud, nos quais o sistema de memória é de alta capacidade, usufruem da utilização de \vms para isolar duplicatas inteiras de \oss com o auxílio da virtualização a nível de instrução~\cite{sharma2016containers}. Em oposição, \lws não dispõem de centenas de GBs de memória, mas sim pequenas memórias locais. Isso associado a outras simplificações de \hardware faz com que algumas técnicas de virtualização sejam impraticáveis nesses ambientes computacionais.


Nesse contexto, visando atenuar o impacto da virtualização no sistema de memória, o presente trabalho explora um modelo de virtualização mais leve, baseado em contêineres adaptado para \lws. O \so executa os contêineres como aplicações virtuais. Sendo assim, não há a necessidade de um \os convidado, resultando em um menor impacto no sistema de memória e requisitando menor complexidade do \hardware~\cite{thalheim2018cntr, sharma2016containers}.


\section{contexto de um processo}
\mytodo{threads, memoria, syscall, comunicação, estruturas de sincronização}
\mytodo{conteinerização no nanvix}



\begin{figure}[t]
	\centering
	
	\subcaptionminipage[fig.nanvix.without-uarea]%
                   {.4\textwidth}
                   {\os sem isolamento.}
                   {\includegraphics[width=\textwidth]{content/images/nanvix-without-uarea-uk.pdf}}
	\qquad
	\subcaptionminipage[fig.nanvix.with-uarea]
                   {.4\textwidth}
                   {\os com isolamento.}
                   {\includegraphics[width=\textwidth]{content/images/nanvix-with-uarea-uk.pdf}}

	\includegraphics[width=.33\linewidth]{content/images/legenda.pdf}
	
	\caption{Diferença da estrutura do \nanvix com e sem a \textit{User Area}.}%
\end{figure}

\subsection{Isolamento do contexto de um processo de usuário}
\label{sec.dev.kernel-usuario}

\subsubsection{Divisão de Dados e Instruções}
\label{sec.divisao-dados-instrucao}

    Para a virtualização de processos através da conteinerização, é recomendável que as informações relevantes para a manipulação dos processos em execução estejam isoladas das informações internas do próprio \os para que os recursos de \hardware sejam utilizados de maneira eficiente~\cite{choudhary2017critical}.
    A \autoref{fig.nanvix.without-uarea} ilustra como os subsistemas do \nanvix são originalmente estruturados. Não há uma divisão explícita do que são dados para funcionamento interno do \os ou dependências locais do processo.
    Esta abordagem torna algumas das funcionalidades do \os onerosas porque ela dificulta o acesso às informações do processo e impacta partes independentes do sistema, \eg migração e segurança dos processos.

    Além disso, a geração original de um executável do \nanvix compila todos os níveis em bibliotecas estáticas (\hal, \microkernel, \libnanvix, \ulibc e \multikernel) e as junta com a aplicação do usuário de forma a misturar o que é \kernel do que é usuário.
    %
    Visando a separação das informações entre usuário e \kernel, nós adaptamos o \script de ligação original do \nanvix. Na nova versão, as seções .text, .data, .bss e .rodata dos arquivos binários compilados são renomeados, especificando qual camada de abstração tal arquivo pertence. Desta forma, é possível identificar dados e instruções de cada camada do \nanvix, assim como as informações do usuário. Sendo assim, são geradas seções .text, .data, .rodata e .bss específicas para o \kernel e usuário. Portanto, todas as informações de \kernel, alocadas nos endereços mais baixos da memória, são isoladas das informações de aplicação, alocadas nos endereços mais altos da memória. Neste processo, são exportadas algumas constantes que apontam onde começam e terminam as partes do binário que são relacionadas ao \kernel e à aplicação. Essas constantes permitem a manipulação e gerenciamento mais precisos dos segmentos de memória do \kernel e da aplicação.
    
    Através dessa estratégia, todos os \clusters passam a ter a mesma organização interna de \kernel, facilitando a migração. Ou seja, a migração pode ser feita através do salvamento dos dados e instruções da aplicação de um \cluster e restauração destes nas respectivas posições em outro \cluster. Essas posições são identificadas pelas constantes exportadas no processo de compilação. Com isso, evita-se manipulações mais complexas do processo como a busca em várias regiões de memória para montar o estado interno do processo.

    % \mytodo{colocar alguma parte do linker?}
    % \mytodo{Souto: ngm vai entender o código do linker mas seria legal colocar e discutir melhor sobre as constantes.}
    
\subsubsection{\textit{User Area}}
\label{sec.uarea}

    Além da separação de dados e instruções entre \kernel e aplicação, é necessário a identificação e separação das estruturas internas do \so que são manipuladas pelo usuário e constituem o estado interno do processo. Nesse contexto, é introduzido o conceito de conteinerização para isolar as dependências que o usuário possui dentro do \cluster. Ou seja, nós isolamos os dados que são gerenciados pelo \kernel mas pertencem ao contexto do processo de usuário. Neste contexto, nós isolamos tais dados em uma região de memória bem definida, denominada de \uarea. 

    Detalhadamente, a \uarea mantém informações sobre:
    \begin{enumerate}[label=(\roman*)]
        \item \Threads ativas, incluindo identificadores e contextos;
        \item Ponteiros para suas pilhas de execução; 
        \item Variáveis de controle e filas de escalonamento;
        \item Estruturas de gerenciamento de chamadas de sistema; e
        \item Estruturas de gerenciamento de memória (\eg sistema de paginação).
    \end{enumerate}

    Essa estrutura genérica foi projetada para englobar as várias arquiteturas suportadas pelo \nanvix. Além disso, a estrutura permite a modificação e expansão, não se limitando ao estado atual do desenvolvimento do \nanvix, para atender os objetivos de outros projetos que usufruam do \nanvix.

\section{Migração de Processos}
\label{sec.migracao}

Como aplicação direta do isolamento do processo, a migração de processos torna-se viável. Especificamente, nós eliminamos a necessidade de descobrir quais são e onde estão as informações que compõem o estado de um processo dentro do \nanvix através da criação de uma instância isolada do espaço do usuário via conteinerização, facilitando a transferência de seu contexto. Isso só é possível porque os \clusters possuem uma estrutura de \kernel idêntica (devido às mudanças desenvolvidas no processo de compilação detalhados na \autoref{sec.divisao-dados-instrucao}). Por este motivo, eliminamos o envio de dados redundantes entre \clusters referentes à instância local do \os, atenuando o impacto da migração sobre a \noc.

\subsection{Rotina de migração}
\label{sec.rotina-migracao}

Para a migração de um processo entre \clusters foi desenvolvida uma rotina de migração. A funcionalidade é similar ao \criu, ferramenta utilizada por \softwares de gerenciamento de contêineres como o Docker. Porém, a migração é executada por intermédio de \daemons do \os. Neste do projeto, implementamos o algoritmo \hotmigration para migração de processos. A técnica de \hotmigration migra a aplicação durante sua execução, copiando as páginas de memória e o estado de execução da aplicação e restaurando a aplicação depois da transferência completa dos dados. A seguir é detalhado o fluxo de execução da migração:

\begin{description}
	\item[1. Congelamento da execução do processo em um estado consistente.] \hfill
	
	Antes do envio da aplicação a outro \cluster, é necessário que o processo esteja em um estado consistente e estático. Isso significa que durante o processo de migração é preciso que todas as operações dele sejam pausadas. Isso é feito objetivando evitar inconsistências que podem ser causadas por condições de corrida \eg impedir perda de instruções, retornos de chamadas de sistemas, sinais de sincronização, etc. Para atingir esse estado consistente, a chamada de sistema \freeze é invocada. Esta é uma chamada de sistema que é tratada apenas pelo \mcore. Especificamente, esta chamada ativa uma variável interna do \so que impede o escalonamento de \threads de aplicação (\threads que não executam no \mcore) e envia um sinal de reescalonamento para todos os \scores, para que as \threads de usuário saiam de execução o mais rápido possível. Isso garante uma pausa na aplicação sem que o \so seja impedido de executar, o que é imprescindível para a migração, já que as informações do processo precisam ser enviadas pelas interfaces \noc do \cluster remetente, o que exige que o \so atenda às requisições de envio de dados. Após o travamento no escalonamento de \threads de usuário, novas chamadas de sistema requisitadas pela aplicação não podem ocorrer. Sendo assim, após a migração, o \cluster destinatário atenderá às chamadas não atendidas e reconhecerá as atendidas, pois as estruturas de sincronização e variáveis de retorno são migradas também durante o processo. Após o congelamento do escalonamento e a retirada das \threads de usuário dos \scores, o processo é considerado consistente e seu contexto está apto para ser migrado.

	\item[2. Transferência do contexto do processo entre \clusters.] \hfill
	
	Com o processo em um estado consistente, uma \task de sistema, que é executada no \mcore, é criada para o envio dos dados ao \cluster destinatário. Através das abstrações de comunicação \mailbox e \portal, as seções de dados e instruções do processo são enviadas ao \cluster destinatário. Logo após, a \uarea é enviada. O envio de dados, instruções e \uarea garantem que o contexto inteiro do processo seja enviado, possibilitando a retomada da execução no \cluster destinatário.

	\item[3. Restauração da execução do processo no \cluster destino.] \hfill
	
	Com o contexto do processo já no \cluster destinatário, a execução é restaurada. Isso é feito pela chamada de sistema \unfreeze, que descongela o escalonamento de \threads de usuário. Assim, a execução do processo continua normalmente, agora em outro \cluster.
\end{description}

\mytodo{detalhar como funcionam as tasks e threads em cada cluster: remetente e destinatario}
\mytodo{como funciona a migração para um cluster com nenhum processo alocado}

\chapter{Resultados Parciais}
\label{chap.results}

A solução foi avaliada em etapas anteriores ao desenvolvimento atual do trabalho e os resultados seguintes englobam apenas o susbsistema de \threads do \nanvix. 
% \todo{Seria bom colocar um capítulo de metodologia com perguntas que gostariamos de responder e detalhamento dos experimentos}
%
Para avaliar o impacto das mudanças feitas para a virtualização, foram desenvolvidos experimentos sobre a manipulação de \threads e suporte à migração de processos no \nanvix. Todos os experimentos foram executados no processador \mppa e os resultados mostrados são valores médios de 100 replicações de cada experimento para garantir 95\% de confiança estatística, resultando em um desvio padrão inferior a 1\%.

\begin{figure}[b]
	\centering
	\subcaptionminipage[fig.fork-join]%
                   {.5\textwidth}
                   {Tempo de criação de \threads.}
                   {\includegraphics[width=\textwidth]{content/images/fork-join-kernel-time-bars.pdf}}
	\qquad
	\subcaptionminipage[fig.kernel-counters]
                   {.4\textwidth}
                   {Métricas do \kernel.}
                   {\includegraphics[width=\textwidth]{content/images/fork-join-kernel-counters.pdf}}
	\caption{Impactos da virtualização sobre a manipulação de \threads.\label{fig.threads}}%
\end{figure}

O experimento de manipulação de \threads mensura os impactos na criação e junção através de diferentes perspectivas. Especificamente, coletamos o tempo de execução, desvios e faltas ocorridas na \cache de dados e de instrução (\autoref{fig.threads}).
Os resultados apresentam um aumento no desempenho das operações de manipulação quando utilizamos a \uarea porque exploramos melhor a localidade espacial dos dados, o que, consequentemente, diminui o número de faltas na \cache.

O experimento de migração avaliou o tempo de transferência de um processo entre \clusters.
A aplicação de usuário migrada contém $352,8$~KB. Detalhadamente, foram transferidos instruções e dados ($342,8$~KB), a \uarea ($2$~KB) e uma pilha de execução ($8$~KB). O \downtime médio da aplicação, \ie o tempo que a aplicação demorou para restaurar a execução no \cluster destinatário após a migração, foi de $226$~ms. A média de tempo para o \cluster remetente enviar todos os dados foi de $218$~ms.
\glsresetall
\chapter{Conclusões}
\label{chap.conclusions}

Neste trabalho foi explorado um modelo de virtualização leve baseada em contêineres que considera as restrições arquiteturais dos \lws, adaptando-se as suas restrições, principalmente relacionadas à memória. A virtualização proposta visa melhorar a mobilidade e gerenciamento de processos para \lws no contexto de em um \os distribuído.
%
Os resultados mostraram que o isolamento das dependências de um processo aumentaram o desempenho de operações do \kernel e suportaram de fato a migração de processos de forma eficiente. Como trabalhos futuros, pretende-se:

\begin{enumerate}[label=(\roman*)]
    \item Ampliar a virtualização, englobando outros subsistemas do \nanvix;
    \item Habilitar a execução simultânea de múltiplas aplicações no processador e sua proteção.
\end{enumerate}
    
    %%%%%%%%%%%%%%%%%%%%%%%%%%%%%%%%%%%%%%%%%%%%%%%%%%%%%%%%%%%%%%%%%%%%
    %%% Elementos pós-textuais                                       %%%
    %%%%%%%%%%%%%%%%%%%%%%%%%%%%%%%%%%%%%%%%%%%%%%%%%%%%%%%%%%%%%%%%%%%%
    
    \postextual
    \bibliography{references}
    
\apendices

\chapter{Artigo Científico}
\label{chap:apendice}

\includepdf[scale=1,pages=-,pagecommand={}]{content/apendices/erad.pdf}

% Uso de \cite em apêndice/anexo como se fossem antes do \bibliography{}.  NBR
% 14724 e NBR 6023, assim como os documentos da BU não especificam nada sobre
% citações dentro de apêndices/anexos. No entanto, em email trocado com a BU, a
% orientação foi de usar \cite{} normalmente e deixar que as referências sejam
% listadas na única bibliografia do documento, mesmo que esta esteja antes dos
% apêndices. A argumentação é que apêndices e anexos são numerados e fazem parte
% do documento, logo suas referências devem ser listadas como referências do
% documento. Além disso as normas não prevem segmentar as referências por
% capítulos.

    
\end{document}
