\glsresetall
\chapter{Trabalhos Relacionados}\label{chap.related-work}

Neste capítulo serão mostrados técnicas e pesquisas que estão sendo desenvolvidas no que diz respeito à virtualização e migração de processos. Serão apresentados alguns trabalhos relacionados, bem como serão evidenciadas as semelhanças e diferenças que o presente trabalho tem em relação àqueles evidenciados.

Grande parte das pesquisas relacionadas a migração estão inseridas em ambientes \cloud. Nesses casos os esforços estão voltados na exploração de várias vantagens que a migração de processos oferece nesses espaços \cite{live-vm-migration-techniques}. Entre elas podem-se citar:
\begin{enumerate}[label=(\roman*)]
    \item Balanceamento de carga;
    \item Tolerência a falhas;
    \item Gerenciamento do consumo de energia;
    \item Compartilhamento de recursos;
    \item Manutenção de sistemas sem interrupções;
\end{enumerate}

Os autores costumam aliar uma ou mais das características citadas acima para desenvolverem suas pesquisas. Por exemplo:
% \begin{enumerate}[label=(\roman*)]
    % \item 
    Aliando-se a transparência de localidade de execução de aplicações com a manutenção destas sem sua suspenção é possível explorar algorítmos ou modelos para melhor posicionar as aplicações na rede com o objetivo de atender a maior quantidade de usuários e de maneira mais eficiente \cite{live-migration-sdn}. Ademais, esses modelos podem ser especializados para tipos específicos de aplicações, como \iot \cite{ada-things};
% \end{enumerate}
\mytodo{linux conteiners, virtualizacao}