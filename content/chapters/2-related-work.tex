\glsresetall
\chapter{Trabalhos Relacionados}
\label{chap.related-work}

Neste capítulo, serão mostradas técnicas e pesquisas que estão sendo desenvolvidas no que diz respeito à virtualização e migração. Serão apresentados trabalhos relacionados, bem como serão evidenciadas as semelhanças e diferenças com o presente trabalho.

Grande parte das pesquisas relacionadas à migração estão inseridas em ambientes \cloud. Nesses casos, os esforços estão voltados para redução do tempo total de migração, diminuição do \downtime~\cite{migration-linux-conteiners,clark2005live} e exploração/otimização das vantagens que a migração de processos oferece nesses ambientes computacionais. Entre essas vantagens podem-se citar:
\begin{enumerate}[label=(\roman*)]
    \item Balanceamento de carga~\cite{live-vm-migration-techniques,ada-things};
    \item Tolerância a falhas~\cite{fernando2019live};
    \item Gerenciamento do consumo de energia~\cite{aldossary2018performance};
    \item Compartilhamento de recursos; e
    \item Manutenção de sistemas sem interrupções~\cite{live-vm-migration-techniques,ada-things}.
\end{enumerate}

Apesar da maioria das pesquisas estarem voltadas à exploração desses benefícios e diminuição do tempo de migração e \downtime em ambientes \cloud, há alguns autores preocupados com o desenvolvimento de soluções envolvendo virtualização e migração em ambientes de recursos restritos. Dessa forma, como a temática de limitação de recursos, especialmente de memória, é muito presente neste trabalho, serão abordados nos próximos parágrafos algumas pesquisas desses autores \ie pesquisas voltadas à busca pelo uso da virtualização/migração de forma mais leve e cujo impacto no \hardware seja reduzido, se adaptando a esses sistemas de recursos limitados.

O artigo \textit{''Virtualization on TrustZone-enabled Microcontrollers? Voilà!''}~\cite{pinto2019virtualization} adressa a possiblidade de implementação da virtualização em microcontroladores que utilizam \trustzone. \trustzone é uma tecnologia de \hardware voltada à segurança, em que a execução de um sistema pode ser dividida entre normal e segura. Os autores afirmam que essa tecnologia pode ser explorada além das suas propriedades de segurança. Isso porque o \trustzone também provê certo nível de isolamento dos recursos, o que o torna viável de ser usado para virtualização, afinal o isolamento cria um ambiente seguro e propício para a execução simultânea e isolada de múltiplas \vms.

O artigo expõe a dificuldade de se implementar a virtualização em \mcus devido aos seus recursos limitados. Nesses ambientes, não é possível a utilização de \hypervisors tradicionais, haja vista a baixa complexidade de \hardware das \mcus. Sendo assim, para atender a necessidade de baixo impacto nos recursos dos \mcus, os autores propõem uma solução que usa um \hypervisor mais leve para gerenciar as \vms nesses ambientes utilizando a tecnologia \trustzone para garantir o isolamento das \vms.

Os testes foram feitos num microcontrolador \textit{Cortex-M4}. Conforme descrito pelos autores, a solução, de fato, garante o suporte à execução múltipla de \vms em microcontroladores.

O artigo \textit{''Checkpointing and migration of IoT edge functions''}~\cite{karhula2019checkpointing} propõe um artifício envolvendo migração de contêiners entre dispositivos \iot de borda como solução para a diminuição do uso de recursos em dispositivos \iot.

Os autores evidenciam que os aparelhos \iot são usados na computação de borda para provomover o que chamamos de \faas, que é um tipo de serviço oferecido por diversas plataformas, como a \textit{Amazon AWS Lambda} e \textit{Google Cloud Functions}. O problema é que esses dispositivos possuem recursos limitados, restringindo-se à execução de poucos contêineres simultaneamente. Além disso, as abordagens tradicionais de \faas sugerem a execução ininterrupta dos contêineres que são iniciados. Isso torna a computação de borda ineficiente, pois esse esquema pode sobrecarregar rapidamente os dispositivos \iot, haja vista a memória limitada desses. A situação se agrava ainda mais quando consideramos funções de longa duração bloqueantes (muito comuns em sistemas de autenticação) \eg funções que esperam alguma requisição, resposta ou qualquer tipo de sinal de outro sistema, seja ele um outro dispositivo \iot ou uma ação humana.

Dessa forma, os autores propõe um esquema de \checkpointing utilizando \docker e \criu. Através dessas tecnologias, os contêineres que não estão executando computação útil são interrompidos e salvos em disco, liberando espaço da memória para a execução de outro contêiner. Isso se torna extremamente útil quando consideramos funções de longa duração bloqueantes, já que durante o tempo de espera pelo sinal, a aplicação pode ser interrompida. Além disso, com o estado salvo em disco, a migração de contêineres entre dispositivos \iot de borda se torna possível. Dessa forma, além de reduzir o uso de recursos nos dispositivos de computação em borda, através da migração dos contêineres, outros benefícios surgem, como o balanceamento de carga e tolerância a falhas entre aparelhos \iot de borda.

Os testes foram feitos em uma \textit{Raspberry Pi 2 Model B}, a qual rodava diversos contêineres com aplicações em \textit{Node JS} de longa duração e que simulavam o comportamento bloqueante. Os resultados apontam que houve economia no uso de recursos, em especial da memória e que a migração de contêineres entre dispositivos \iot de borda é possível.


% Os autores costumam aliar uma ou mais das características citadas acima para desenvolverem suas pesquisas. Por exemplo:

%     Para evitar que muitos usuários tenham que fazer requisições à aplicações hospedadas em servidores distantes ou com muito tráfego, pode-se utilizar a transparência de localidade de execução de aplicações com a manutenção destas sem sua suspenção. Assim, é possível explorar algorítmos ou modelos para melhor posicionar as aplicações na rede com o objetivo de atender a maior quantidade de usuários, de maneira mais eficiente e sem que o sistema precise ser desligado~\cite{live-migration-sdn}. Ademais, esses modelos podem ser especializados para tipos específicos de aplicações, como \iot~\cite{ada-things}.

%     Além disso, pesquisas são feitas com o intuito de tornar a migração de \vms entre servidores tolerante a falhas~\cite{fernando2019live}. A migração de \vms por algum motivo pode falhar (devido a problemas de rede, por exemplo). Nesse contexto, alguns autores sugerem o uso de \checkpoints, os quais são estados consistentes da \vm em dado momento. Se a migração falhar, a \vm pode ser restaurada para um \checkpoint. Isso evita que a \vm seja perdida.

% Em contraste com os trabalhos apresentados, nossa proposta não está inserida em ambientes \cloud. Seu foco está na migração de processos entre \clusters em um mesmo \chip, em um sistema computacional restritivo que restringe as técnicas possíveis a serem utilizadas. Além disso, nosso trabalho se difere dos demais trabalhos por explorar a virtualização e migração de processos inserida no contexto dos \lws e não apenas buscando otimizar algoritmos específicos, \eg \livemigration.